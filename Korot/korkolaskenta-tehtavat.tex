\documentclass[a4paper,10pt]{article}\usepackage[]{graphicx}\usepackage[]{color}
%% maxwidth is the original width if it is less than linewidth
%% otherwise use linewidth (to make sure the graphics do not exceed the margin)
\makeatletter
\def\maxwidth{ %
  \ifdim\Gin@nat@width>\linewidth
    \linewidth
  \else
    \Gin@nat@width
  \fi
}
\makeatother

\definecolor{fgcolor}{rgb}{0.345, 0.345, 0.345}
\newcommand{\hlnum}[1]{\textcolor[rgb]{0.686,0.059,0.569}{#1}}%
\newcommand{\hlstr}[1]{\textcolor[rgb]{0.192,0.494,0.8}{#1}}%
\newcommand{\hlcom}[1]{\textcolor[rgb]{0.678,0.584,0.686}{\textit{#1}}}%
\newcommand{\hlopt}[1]{\textcolor[rgb]{0,0,0}{#1}}%
\newcommand{\hlstd}[1]{\textcolor[rgb]{0.345,0.345,0.345}{#1}}%
\newcommand{\hlkwa}[1]{\textcolor[rgb]{0.161,0.373,0.58}{\textbf{#1}}}%
\newcommand{\hlkwb}[1]{\textcolor[rgb]{0.69,0.353,0.396}{#1}}%
\newcommand{\hlkwc}[1]{\textcolor[rgb]{0.333,0.667,0.333}{#1}}%
\newcommand{\hlkwd}[1]{\textcolor[rgb]{0.737,0.353,0.396}{\textbf{#1}}}%

\usepackage{framed}
\makeatletter
\newenvironment{kframe}{%
 \def\at@end@of@kframe{}%
 \ifinner\ifhmode%
  \def\at@end@of@kframe{\end{minipage}}%
  \begin{minipage}{\columnwidth}%
 \fi\fi%
 \def\FrameCommand##1{\hskip\@totalleftmargin \hskip-\fboxsep
 \colorbox{shadecolor}{##1}\hskip-\fboxsep
     % There is no \\@totalrightmargin, so:
     \hskip-\linewidth \hskip-\@totalleftmargin \hskip\columnwidth}%
 \MakeFramed {\advance\hsize-\width
   \@totalleftmargin\z@ \linewidth\hsize
   \@setminipage}}%
 {\par\unskip\endMakeFramed%
 \at@end@of@kframe}
\makeatother

\definecolor{shadecolor}{rgb}{.97, .97, .97}
\definecolor{messagecolor}{rgb}{0, 0, 0}
\definecolor{warningcolor}{rgb}{1, 0, 1}
\definecolor{errorcolor}{rgb}{1, 0, 0}
\newenvironment{knitrout}{}{} % an empty environment to be redefined in TeX

\usepackage{alltt}


\usepackage[utf8]{inputenc}
\usepackage[T1]{fontenc}
\usepackage[finnish]{babel}
\usepackage{amsthm}
\usepackage{amsmath}
\usepackage{amssymb}
\usepackage{mathtools}
\usepackage[math]{kurier}
\usepackage{eurosym}
\usepackage{caption}
\usepackage{enumerate}
\usepackage{geometry}
% package for exercises
\usepackage{exsheets}
\DeclareTranslation{finnish}{exsheets-exercise-name}{Tehtävä}
\DeclareTranslation{finnish}{exsheets-question-name}{Kysymys}
\DeclareTranslation{finnish}{exsheets-solution-name}{Ratkaisuehdotus, t.}
%\newtheorem{teht}{Tehtävä}
%\theoremstyle{remark}
%\newtheorem{ratk}{Ratkaisuehdotus}
% package for captions without numbers
\usepackage{caption}

% package for exercises
\usepackage{exsheets}
\DeclareTranslation{finnish}{exsheets-exercise-name}{Tehtävä}
\DeclareTranslation{finnish}{exsheets-question-name}{Kysymys}
\DeclareTranslation{finnish}{exsheets-solution-name}{Ratkaisuehdotus, t.}

%\SetupExSheets{solution/print=true}
\IfFileExists{upquote.sty}{\usepackage{upquote}}{}
\begin{document}



\pagestyle{empty}

\section*{Luentotehtäviä}

\subsection*{Korkolaskenta}

Huomioita
\begin{itemize}
    \item Lähdeveroa tai varsinkaan sen pyöristyssääntöä ei tarvitse ottaa tehtävissä huomioon, ellei toisin mainita.
    \item Jos yksinkertaisen korkolaskun tehtävissä on selvitettävä päivien lukumäärä, niin käytä yleistä laskutapaa todelliset/360, ellei toisin mainita.
\end{itemize}

\subsubsection*{Yksinkertainen korkolasku}

\begin{question} Mikä tilin korkokannan tulisi olla, jotta sinne talletettu 1000 euroa kasvaisi neljässä kuukaudessa korkoa 3,33 euroa?
\end{question}\begin{solution}
    \[
    r = kit = 1000\cdot i\cdot \frac{4}{12} = 3,33\Leftrightarrow i = \frac{3,33}{1000\cdot\frac{4}{12}}\approx 0,01
    \]
    Siis korkokannan tulee olla 1~\%.
\end{solution}

\begin{question} Rafaelin tilin korkokanta oli kiinteä 1,3~\%. Hänen tallettamansa summa kasvoi yhdeksässä kuukaudessa korkoa 1,95 \euro.
Kuinka suuren summan hän oli alunperin tallettanut tililleen?
\end{question}\begin{solution}
    \[
    r = kit \Leftrightarrow k = \frac{r}{it} = \frac{1,95}{0,013\cdot\frac{9}{12}} = 200
    \]
    Alkuperäinen pääoma on 200 euroa.
\end{solution}

\begin{question} Henkilö ottaa pakon edessä 500~\euro\ lainaa korkokannan ollessa 15~\% p.a. Kuinka pian velan suuruus on 550 euroa?
\end{question}\begin{solution}
    \[
        r = kit \Leftrightarrow t = \frac{r}{ki} = \frac{50}{500\cdot0,15} = \frac{2}{3}
    \]
    Korkoajan on oltava 2/3-vuotta, eli 8 kuukautta.
\end{solution}

\begin{question} Ronja talletti 314 euroa tammikuun alussa tililleen. Hän nosti rahat kesäkuun lopussa, jolloin nettokorkoa oli kertynyt
 1,70 euroa. Mikä oli tilin nettokorkokanta? Entä korkokanta?
\end{question}\begin{solution}
    Korkoa maksetaan kuudelta kuukaudelta. Nyt
    \[
        i = \frac{r}{kt} = \frac{1,70}{314\cdot\frac{6}{12}}\approx 0,01083
    \]
    joten nettokorkokanta  on likimain 1,1~\%. Korkokanta on \(\frac{0,01083}{0,7}\approx 1,5~\%\).
\end{solution}

\begin{question} Anja nosti talletuksensa vuoden kuluttua talletushetkestä. Hän sai pankista verojen vähentämisen jälkeen 1073,20 euroa.
 Kuinka suuren summan hän oli tililleen tallettanut, kun tilin korkokanta oli 1,1~\% ja lähdevero 30~\%?
\end{question}
\begin{solution}
    \[
        K = k + r = k + kit = k(1+it)
    \]
    Nyt \(t = 1\), joten
    \[
        1073,20 = k(1+0,7\cdot0,011)\Leftrightarrow k = \frac{1073,20}{1+0,7\cdot0,011} \approx 1064,9995
    \]
    Alkuperäinen pääoma on siis 1065,00 euroa.
\end{solution}

\begin{question}
        Mikä pääoma tuottaa neljässä kuukaudessa nettokorkoa 150 euroa, jos korkokanta on 3{,}6~\%~p.s.? Muista, että lähdevero on 30 \%.
\end{question}
\begin{solution}
    Olkoon \(k\) tuntematon pääoma. Nettokorkokanta on \(0,70\cdot0{,}036 = 0{,}0252\).
    Korkokausi on puoli vuotta ja korkoaika neljä kuukautta, eli kaksi kolmasosaa korkokaudesta.
    Kaavan mukaan on oltava
    \[
        150 = k\cdot 0{,}0252\cdot \frac{4}{6} = k\cdot 0{,}0168,
    \]
        joten
    \[
        k = \frac{150}{0{,}0168} = 8~928{,}571
    \]
    euroa. Näin ollen 8~928{,}57 euroa ei aivan riitä, joten tilille täytyy tallettaa ainakin 8~928{,}58 euroa. (Tosiasiassa lähdeveron pyöristyssäännön seurauksena vähempikin riittäisi.)
\end{solution}

\begin{question} 
Saat yllättäen 75~000 euroa ja talletat koko summan 3.7.2014 pankkitilille, jonka korkokanta on 2{,}8~\%. Nostat rahat tililtä 19.11.2014. kuinka paljon saat nettokorkoa?
\end{question}
\begin{solution}
    korkopäiviä kertyy \(31-3+31+30+31+19 = 139\). nettokorko on
    \[
        r = kit = 75000\cdot0,7\cdot 0,028\cdot\frac{139}{360}\approx 567,58
    \]
    euroa.
\end{solution}

\begin{question}
   Talletat huhtikuun 5. päivä alussa 1~410 euroa tilille, jonka korkokanta on 2{,}6~\%~p.a. Milloin tililtä on nostettavissa 1415 euroa? Huomioi 30~\% lähdevero. Käytä korkoajan laskemiseen tapaa ``todelliset/360''.
\end{question}
\begin{solution}
    Tililtä on nostettavissa 1415 euroa viimeistään silloin, kun pääoma on kasvanut ``bruttokorkoa'' \(5/0,7 = 7{,}142857\) euroa. Jos 
    \[
        7{,}142857 = 1~410\cdot0{,}026\cdot\frac{t}{360}
    \]
    niin
    \[
        t = \frac{7{,}142857\cdot360}{1~410\cdot0{,}026} = 70{,}14262
    \]
     Aikaa tarvitaan siis vähintään 71 päivää, kun lähdeveron pyöristyssääntöä ei huomioida.
    % 70 päivältä korkoa kertyisi
    % \[
    %     r = k\cdoti\cdot\frac{70}{360} = r <- k*i*70/360; r
    % \]
    % euroa, josta veroa menisi laskennallisesti \(0,3\cdotr = vero <- 0.3*r; vero\) ja pyöristyksen jälkeen \(todvero <- floor(10*vero)/10; todvero\) euroa. Korkoa maksetaan 70 päivältä \(r - todvero = r-todvero\) euroa, joten 70 päivää riittää. 
    Koska talletuspäivää ei lasketa korkopäiviin, mutta korkopäivä lasketaan, rahat olisivat nostettavissa kesäkuun 15. päivä.
\end{solution}

\begin{question} 
    Kaverisi säästää rahaa pitkää lomareissua varten tallettamalla joka kuukauden alussa yhtä suuren summan rahaa tililleen.
Kaverisi aloittaa säästämisen tammikuun alussa ja haluaa nostaa rahat vuoden kuluttua.
Kuinka paljon rahaa hänen tulisi tallettaa tililleen kuukausittain, jotta vuoden kuluttua nostettavissa olisi 1 200 euroa? Tilin vuosikorko on 2{,}5~\%. Muista lähdevero 30~\%.
\end{question}\begin{solution}
    Olkoon \(k\) kuukausittain talletettava pääoma. Tammikuun talletus kasvaa korkoa 12 kk, helmikuun talletus 11 kk jne.
    \begin{align*}
        r &= k\cdot0,7\cdot0,025\cdot\left(\frac{12}{12} + \frac{11}{12} + \ldots + \frac{2}{12} + \frac{1}{12}\right)\\
         =& k\cdot0,7\cdot0,025\cdot\left(12\cdot\frac{\frac{12}{12} + \frac{1}{12}}{2}\right)\\
         =& k\cdot0,7\cdot0,025\cdot\frac{78}{12}\\
         =& 0,11375\cdot k
    \end{align*}
    Nyt \(1200 = 12k + r = 12k + 0,11375k = 12{,}11375k\) eli \(k = 1200/12{,}11375\approx 99,061\).
    Tilille on siis talletettava vähintään 99,07 euroa.
\end{solution}

\subsubsection*{Koronkorko}

\begin{question} 
    Sijoitat 500 euroa sijoitusvakuutukseen, joka lupaa vähintään 3{,}0 \% vuotuisen koron sijoitukselle.
Neljän vuoden kuluttua sijoituksen arvo oli kasvanut 60 euroa. Toteutuiko sijoitusvakuutuksen lupaama vuosikorko?
\end{question}\begin{solution}
    Lupauksen mukaan tililtä olisi nostettavissa \(500\cdot1,03^4\approx562,75\) euroa. Lupaus ei siis toteutunut.
\end{solution}

\begin{question} 
    Viivi tallettaa saamansa 1800 euron palkan tilille, jonka korkokanta 2,1 \%. Kuinka suuren summan Viivi voi nostaa tililtä
neljän vuoden kuluttua? Muista lähdevero.
\end{question}
\begin{solution}
    Nettokorkokanta on \(0,7\cdot0,021 = 0,0147\),
    joten neljän vuoden kuluttua summa on
    \[
        1800\cdot1,0147^4\approx1908,1967\dots
    \]
    euroa. Tililtä on siis nostettavissa 1908,19 euroa.
\end{solution}

\begin{question} 
    Mikä tulisi Viivin tilin korkokannan olla edellisen tehtävän tilanteessa, jotta tililtä olisi nostettavissa 2100 euroa?
\end{question}\begin{solution}
    \[
        2100 = 1800\cdot q^4\Leftrightarrow q^4 = \frac{2100}{1800}
    \]
    Tästä saadaan \(q = 1,0393\dots\), joten nettokorkokannan tulisi olla 3,9~\% ja korkokannan 5,6~\%.
\end{solution}

\begin{question} 
    Kaukainen sukulainen pyytää sinua avaamaan tilin, jotta hän voi jokaisen vuoden lopussa lahjoittaa sinulle tietyn
summan rahaa tallettamalla sen tälle tilille. Tilillä ei ole näiden vuosittaisten talletusten ja koronmaksun lisäksi muita tilitapahtumia.
Ensimmäisen talletuksen sukulaisesi tekee vuoden 2014 lopussa. Viiden vuoden kuluttua vuoden 2020 alussa tilillä on rahaa 4~342{,}15 euroa.
Kuinka suuri on vuotuinen talletus, jos tilin korko on 1{,}9 \% p.a? Huomioi lähdevero.
\end{question}
\begin{solution}
    Olkoon \(x\) vuosittain talletettava (lahjoitettava) summa. Vuonna 2014 saatu summa kasvaa korkoa 5 kertaa,
    vuonna 2015 saatu summa 4 kertaa jne., vuonna 2019 saatu summa ei ehdi kasvaa korkoa kertaakaan
    (sillä talletus tehdään vuoden lopussa). Korkoa kasvaneiden talletusten summa on
    \begin{align*}
        &x\cdot q^5 + x\cdot q^4+ x\cdot q^3 + x\cdot q^2+ x\cdot q + x \\
        =& x + x\cdot q+ x\cdot q^2 + x\cdot q^3+ x\cdot q^4 + x\cdot q^5\\
        =& x\cdot\frac{1-q^6}{1-q}
    \end{align*}
    joten
    \[
       x\cdot\frac{1-q^6}{1-q} = 4324,15
    \]
    Tästä saadaan
    \[
        x = \frac{4342,15\cdot(1-q)}{1-q^6}
    \]
    Sijoitetaan \(q= 1 + 0,7\cdot0,019=1,0133\) ja saadaan
    \[
    x = \frac{4342,15\cdot(1-1,0133)}{1-1,0133^6} \approx 699,99979
    \]
    Vuosittainen talletus on siis 700,00 euroa.
\end{solution}

\begin{question} 
    Uuden vuoden lupauksena aloitat säästämisen ja talletat tammikuun lopusta alkaen joka toisen kuukauden lopussa 300 euroa tilille,
jonka korkokanta on 2{,}1 \% p.a. Kuinka paljon rahaa tililläsi on neljännen talletusvuoden lopussa koronmaksun jälkeen? Huomioi lähdevero.
\end{question}
\begin{solution}
    Tarkastellaan ensin yhden vuoden talletuksia. Tammikuun talletus kasvaa korkoa 11 kuukautta, maaliskuun talletus 9 kuukautta,
    toukokuun talletus 7 kuukautta jne., marraskuun talletus kasvaa korkoa yhden kuukauden. Vuoden talletuksille korkoa maksetaan yhteensä
    \begin{align*}
        &300\cdot0,7\cdot0,021\cdot\left(\frac{11}{12} + \frac{9}{12} + \frac{7}{12} + \frac{5}{12} + \frac{3}{12} + \frac{1}{12}\right)\\
        =&300\cdot0,7\cdot0,021\cdot\frac{1}{12}\cdot(11+9+7+5+3+1)\\
        =& 13,23
    \end{align*}
    euroa. Tilannetta voidaan ajatella siis siten, että kunkin vuoden \emph{lopussa} tilille talletetaan \(6\cdot300+13,23 = 1~813{,}23\) euroa.
     Kaiken kaikkiaan tilillä on neljännen talletusvuoden lopussa rahaa
     \[
         K = 1~813{,}23\cdot q^3 + 1~813{,}23\cdot q^2 + 1~813{,}23\cdot q + 1~813{,}23 = 1~813{,}23\cdot\frac{1-q^4}{1-q}
     \]
    euroa. Sijoitetaan \(q = 1 + 0,7\cdot0,021 = 1,0147\) ja saadaan
    \[
        K = 1~813{,}23\cdot\frac{1-1,0147^4}{1-1,0147} =  7~414{,}42
    \]
    Tililtä on siis nostettavissa 7~414{,}42 euroa.
\end{solution}

\subsubsection*{Diskonttaus}

\begin{question} Josefiina tallettaa tililleen vuoden alussa summan, jonka toivoo kasvavan ainakin 5000 euroksi kymmenessä vuodessa. Tilin nettokorkokanta on 1,5~\%.
\begin{enumerate}[(a)]
    \item Selvitä diskonttaustekijä.
    \item Kuinka suuri summa Josefiinan tulee tililleen tallettaa?
\end{enumerate}
\end{question}
\begin{solution}
    \begin{enumerate}[(a)]
        \item Koska \(q = 1 + 0,015 = 1,015\), diskonttaustekijä on \(q^{-10} = \frac{1}{q^{10}}\approx 0{,}8616672\).
        \item \(k = K\cdot q^{-10} = K\cdot 0{,}8616672 = 5000\cdot 0{,}8616672\approx 4~308{,}336\) euroa.
        Josefiinan tulee tallettaa tililleen vähintään 4~308{,}34 euroa.
    \end{enumerate}
\end{solution}

\begin{question} 
Kauppias tarjoaa auton maksamiseen seuraavaa rahoitusmallia: käsiraha 8~000~\euro \ on maksettava heti ja tämän
jälkeen kahtena seuraavana vuonna maksetaan 5 000~\euro. Kolmen vuoden kuluttua maksetaan vielä 2 000~\euro.
Jos asiakas kuitenkin haluaa maksaa auton käteisellä heti ostohetkellä, mikä auton hinnan pitäisi olla, että asiakas
maksaisi siitä yhtä paljon kuin kauppiaan ehdottamassa rahoitusmallissa? Oletetaan, että kauppiaan käyttämä korkokanta on 7{,}5~\%.
\end{question}
\begin{solution}
    Diskontataan maksuerät nykyhetkeen. Kerroin on \(q= 1+0,075 = 1,075\), joten nykyarvoksi saadaan
    \[
        8000 + \frac{5000}{1,075} + \frac{5000}{1,075^2} + \frac{2000}{1,075^3} = 18587,74699
    \]
    euroa. Auton hinta ostohetkellä tulisi olla siis 18587,75 euroa.
\end{solution}


\begin{question} Kuinka paljon lukion stipendirahastoon on lahjoitettava rahaa eurona, kun tarkoituksena on jakaa lahjoitus korkoineen
stipendeinä seuraavasti: tasan vuoden kuluttua lahjoituksesta 200 euroa, kahden vuoden kuluttua 300 euroa,
kolmen vuoden 400 euroa, neljän 500 euroa ja viiden vuoden kuluttua 600 euroa. Pääomaan lisätään vuosittain 4,5 prosentin korko,
ensimmäisen kerran vuoden kuluttua lahjoituksesta. [S2001, 14]
\end{question}\begin{solution}
    Diskontataan maksuerät nykyhetkeen. Kerroin on \(q= 1+0,045 = 1,045\), joten nykyarvoksi saadaan
    \[
        \frac{200}{1,045} + \frac{300}{1,045^2} + \frac{400}{1,045^3} + \frac{500}{1,045^4} + \frac{600}{1,045^5} \approx 1717,38
    \]
    euroa.
\end{solution}

\begin{question} Ostat kalliin vekottimen, josta maksat kaupantekohetkellä 400 euron käsirahan ja kuukauden kuluttua 800 euroa korkokannan ollessa 6{,}6 \% p.a.
Kaverisi osti samanlaisen vekottimen, maksoi ostohetkellä 300 euroa ja neljän kuukauden kuluttua loput.
Kuinka paljon kaverisi maksoi toisessa maksuerässä, jos vekottimenne olivat yhtä kalliit?
\end{question}\begin{solution}
    Oman hinnan nykyarvo on
    \[
        400 + \frac{800}{1+0,066\cdot\frac{1}{12}} = 400 + \frac{800}{1,0055} = 1195,624
    \]
    euroa. Kaverin hinnan nykyarvo
    \[
        300 + \frac{x}{1+0,066\cdot\frac{4}{12}} = 300 + \frac{x}{1,022}
    \]
    Jos hinnat ovat samat, niin
    \[
        300 + \frac{x}{1,022} =  1195,624
    \]
    eli \( x = 1,022\cdot(1195,624-300)\approx 915,32\) euroa.
\end{solution}

\begin{question} Tuttavasi on myymässä asuntoaan ja saa kaksi ostotarjousta. Ensimmäisessä tarjouksessa luvataan maksaa heti 145 000 euroa.
Toisessa tarjouksessa luvataan maksaa heti 64 000 euroa, vuoden kuluttua 54 000 euroa ja kahden vuoden kuluttua 30 000 euroa.
Kumpi tarjous on tuttavasi kannalta parempi, jos korkokanta on 4 \%? Entä, jos korkokanta on 1~\%?
\end{question}\begin{solution}
    Diskontataan toisen tarjouksen summat nykyhetkeen. Jos korkokanta on 4~\%, niin nykyarvo on
    \[
        64000 + \frac{54000}{1,04} + \frac{30000}{1,04^2} \approx 14365,98
    \] euroa.
    Jos korkokanta on 1~\%, niin nykyarvo on vastaavasti
    \[
        64000 + \frac{54000}{1,01} + \frac{30000}{1,01^2} \approx 14687,42
    \] euroa.
    Näin ollen 4~\% korkokannalla kertamaksu on tuttavalle parempi, 1~\% korkokannalla osamaksu.
\end{solution}

\begin{question} Yritys on hankkimassa uuden tuotantohärvelin. Kauppias tarjoaa kaksi maksuvaihtoehtoa:
koko kauppahinta 15 000 euroa kaupanteon yhteydessä tai 7 000 euroa kaupanteon yhteydessä,
5 000 euroa vuoden kuluttua ja 3 500 euroa kahden vuoden kuluttua. Jos yritys maksaa investoinnin lainalla,
jonka vuotuinen korkokanta on 4{,}5 \%, kumpi maksuvaihtoehto sen kannattaa valita?
\end{question}\begin{solution}
    Diskontataan toisen tarjouksen summat nykyhetkeen. Korkokanta on 4,5~\%, joten erien yhteenlaskettu nykyarvo on
    \[
        7000 + \frac{5000}{1,0045} + \frac{3500}{1,0045^2} \approx 14989,74
    \]
    euroa. Osamaksu tulee siis yritykselle halvemmaksi.
\end{solution}

\newpage
\subsection*{Ratkaisuehdotuksia}
\printsolutions

\end{document}
