\documentclass[a4paper,10pt]{article}
\input{../../_texPartials/ekonomityylitLuentotehtavat.sty}
\usepackage{eurosym}
\usepackage{caption}
\begin{document}
\pagestyle{empty}

\section*{Verotehtäviä}

\begin{enumerate}
\subsubsection*{Arvonlisävero}
\item Tuotteen alv:ton hinta on 10,50 euroa. Laske myyntihinta, kun kyseessä on (a) kirja (alv 10\%) (b) elintarvike (alv 14\%).

\item Ostosten loppusummaksi tuli 25,50 euroa. Kuitin mukaan hinta sisältää 3,13 arvonlisäveroa. Selvitä, minkä alv-kannan piiriin tuote kuuluu.

\item  Ostokset maksoivat 76,50\euro. Laske alv:ton hinta, kun ostokset kuuluvat (a) 14\% (b) 24\% arvonlisäveron piiriin.

\item Kengät (alv 24\%) maksoivat 115,00\euro\ ja elokuvalippu (alv 10\%) 10,50\euro. Kuinka paljon arvolisäveroa maksettiin yhteensä?

\item Makeissekoituksen veroton hinta on 15,45 euroa/kg. Makeisveron (hinnanlisä) suuruus on 95 senttiä kilogrammalta. Laske 500g karkkipussin myyntihinta (sis. alv 14\%).

\item Kioskilta ostettiin karkkipussi (alv 14\%) ja aikakauslehti (alv 24\%). Loppusumma 9,60 \euro\ sisälsi kuitin mukaan arvonlisäveroa 1,61 euroa. Laske karkkipussin ja aikakauslehden alv:ttomat hinnat.

\item Leipomo valmistaa leipiä, jotka se myy tukkukauppiaalle. Leipomo osti leivontaan tarvittavia elintarvikkeita 7800 eurolla ja myi valmiita leipiä 15600 eurolla. Molemmat hinnat sisältävät 14\% alv. Kuinka paljon arvonlisäveroa leipomo tilittää  valtiolle?


\item Akseli myy visakoivusta veistämiään puukonkahvoja Ilmariselle hintaan 9 \euro/kpl. Ilmarinen jatkaa työn loppuun ja myy valmiit puukot Elinalle 21 euron kappalehintaan. Elina myy puukkoja myymälänsä asiakkaille hintaan 32 \euro/kpl. Kuinka paljon Elinan asiakas maksaa arvonlisäveroa puukosta? Laske myös, kuinka paljon kukin myyntiketjun jäsen (Akseli, Ilmari, Elina) tilittää arvonlisäveroa valtiolle. Käytä yleistä 24\% verokantaa.


\subsubsection*{Tuloverotus}
\emph{Tehtävissä tarvittavat tuloveroasteikot löydät tämän osion lopusta.}

\item Henkilön verokorttiin on sekä päätuloa, että sivutuloa varten merkitty perusprosentiksi 14,5 ja lisäprosentiksi 32,5 tulorajan 620\euro/kk ylittävältä osalta. Laske nettopalkka, kun kuukausitulot ovat (a) 600,00 euroa (b) 724,80 euroa. Palkasta pidätetään veroennakon lisäksi 6,35\% työeläke- ja työttömyysvakuutusmaksuja.

\item Työntekijän tilille maksettava nettopalkka oli 1300\euro/kk. Verokortin perusprosentti oli 14,5 ja lisäprosentti 32,0. Tulorajaksi oli merkitty 1500\euro/kk. Veroennakon lisäksi palkasta perittiin 6,35\% työeläke- ja työttömyysvakuutusmaksuja. Laske työntekijän bruttopalkka.

\item Henkilön palkasta jäi ennakonpidätyksen jälkeen käteen 73\%. Vuodenvaihteessa henkilö sai 45 euron palkankorotuksen. Samanaikaisesti ennakonpidätys laski 0,8 prosenttiyksikköä. Henkilölle jäi ennakonpidätyksen jälkeen nyt 49,60 euroa enemmän kuin ennen muutoksia. Mikä oli henkilön uusi ennakonpidätysprosentti, ja mikä oli hänen uusi palkkansa?

\item Työntekijän verotettava ansiotulo valtion tuloverotuksessa vuonna 2013 oli 32 531 euroa. Laske veron suuruus.

\item Ronja maksoi vuonna 2013 valtion tuloveroa 3050 euroa. Laske hänen verotettava ansiotulonsa valtion tuloverotuksessa.

\item Helsinkiläisen, kirkkoon kuulumattoman Emilian vuosituloiksi vuonna 2015 arvioitiin \(37 000\) euroa ennakonpidätyprosenttia laskettaessa. Vähennyksiksi valtionverotuksessa arvioitiin 800 euroa ja kunnallisverotuksessa 1250 euroa. Mikä tulee ennakonpidätysprosentiksi? Helsigin kunnallisveroprosentti on 18,5. Yle-veron ja sairausvakuutusmaksun voi jättää huomiotta.

\item Asuntosijoittaja Arffman vuokrasi omistamansa kaksion 820 euron kuukausivuokralla vuodeksi. Asunnossa tehtiin vuokra-aikana pintaremontti, jonka hinnaksi tuli 510\euro. Taloyhtiön perimä yhtiövastike oli 142\euro/kk ja asuntoon kohdistuvan lainan korkoja maksettiin yhteensä 250\euro. Laske pääomatulovero ja asunnon vuokraamisesta saatu nettotulo.

\item Osakesijoittaja myy 500 kpl erään yhtiön osakkeita hintaan 35 000\euro. Hän maksaa osakkeiden myynnistä välityspalkkiota 1,10 \euro/kpl. Aikoinaan sijoittaja maksoi osakkeista 22 000 euroa ja maksoi välityspalkkiota 85 snt/kpl. Laske luovutusvoitto ja pääomatuloveron suuruus.

\centering
	\begin{tabular}{|lrc|}
        \hline
        Tulo (\euro)   & Vero alarajalla  & Veroprosentti \\
        \hline
        0 -- 30 000 & 0 &  30            \\
        30 000 --   & 9000 & 33   \\
        \hline
	\end{tabular}
\begin{table}[h]
    \centering
    \begin{tabular}{|lrc|}
    \hline
    Verotettava ansiotulo & Vero alarajalla & Veroprosentti \\
    \hline
    16 500—24 700                & 8                             & 6,5                                        \\
    24 700—40 300                & 541                           & 17,5                                       \\
    40 300—71 400                & 3 271                         & 21,5                                       \\
    71 400—90 000                & 9 957,50                      & 29,75                                      \\
    90 000—                      & 15 491                        & 31,75                                     \\
    \hline
    \end{tabular}
    \caption*{Tuloveroasteikko 2015}
\end{table}
    \centering
    \begin{tabular}{|lrc|}
    \hline
    Verotettava ansiotulo & Vero alarajalla & Veroprosentti \\
    \hline
    16 100—23 900                & 8                             & 6,5                                        \\
    23 900—39 100                & 515                           & 17,5                                       \\
    39 100—70 300                & 3 175                         & 21,5                                       \\
    70 300—100 000                & 9 883                      & 29,75                                      \\
    100 000—                      & 18 718,75                        & 31,75                                     \\
    \hline
    \end{tabular}


\subsubsection*{Perintö- ja lahjavero}
\item Isä haluaa lahjoittaa pojalleen 18 000 \euro. Kuinka suuri on lahjavero, kun (a) lahja annetaan yhdellä kertaa (b) kahdessa 9000 euron erässä, ja lahjoitusten väli on yli kolme vuotta.

\item Henkilö saa tädiltään perinnön, josta hän maksaa veroja 7729\euro. Kuinka suuresta perinnöstä on kyse?
\end{enumerate}
\subsubsection*{Vastauksia}
\begin{enumerate}
\item (a) 11,55 euroa (b) 11,97 euroa
\item 14\%
\item (a) 67,11 euroa (b) 61,69 euroa
\item 23,21 euroa
\item 9,35 euroa
%bussilippu ja matkaeväät
\item 3,08 euroa ja 4,91 euroa
\item 957,90 euroa
%visakoivuiset puukot
\item Yhteensä alvia maksetaan 6,19\euro; Akseli: 1,74\euro, Ilmarinen: 2,32\euro, Elina: 2,13\euro
\item (a) 474,90 euroa (b) 554,82 euroa
\item 1682,89 euroa
\item 26,2\%, 2093,75 euroa
\item 2025,43 euroa
\item 38385,71 euroa
\item 24,8\%
\item Vero 2212,80 euroa, nettotulo 5163,20 euroa
\item Luovutusvoitto 12025 euroa, vero 3607,5 euroa
\item (a) 1250 euroa (b) 1000 euroa (vuoden 2015 verotus)
\item 52700 euroa
\end{enumerate}

\end{document}