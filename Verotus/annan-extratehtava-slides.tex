\documentclass[]{beamer}\usepackage[]{graphicx}\usepackage[]{color}
%% maxwidth is the original width if it is less than linewidth
%% otherwise use linewidth (to make sure the graphics do not exceed the margin)
\makeatletter
\def\maxwidth{ %
  \ifdim\Gin@nat@width>\linewidth
    \linewidth
  \else
    \Gin@nat@width
  \fi
}
\makeatother

\definecolor{fgcolor}{rgb}{0.345, 0.345, 0.345}
\newcommand{\hlnum}[1]{\textcolor[rgb]{0.686,0.059,0.569}{#1}}%
\newcommand{\hlstr}[1]{\textcolor[rgb]{0.192,0.494,0.8}{#1}}%
\newcommand{\hlcom}[1]{\textcolor[rgb]{0.678,0.584,0.686}{\textit{#1}}}%
\newcommand{\hlopt}[1]{\textcolor[rgb]{0,0,0}{#1}}%
\newcommand{\hlstd}[1]{\textcolor[rgb]{0.345,0.345,0.345}{#1}}%
\newcommand{\hlkwa}[1]{\textcolor[rgb]{0.161,0.373,0.58}{\textbf{#1}}}%
\newcommand{\hlkwb}[1]{\textcolor[rgb]{0.69,0.353,0.396}{#1}}%
\newcommand{\hlkwc}[1]{\textcolor[rgb]{0.333,0.667,0.333}{#1}}%
\newcommand{\hlkwd}[1]{\textcolor[rgb]{0.737,0.353,0.396}{\textbf{#1}}}%

\usepackage{framed}
\makeatletter
\newenvironment{kframe}{%
 \def\at@end@of@kframe{}%
 \ifinner\ifhmode%
  \def\at@end@of@kframe{\end{minipage}}%
  \begin{minipage}{\columnwidth}%
 \fi\fi%
 \def\FrameCommand##1{\hskip\@totalleftmargin \hskip-\fboxsep
 \colorbox{shadecolor}{##1}\hskip-\fboxsep
     % There is no \\@totalrightmargin, so:
     \hskip-\linewidth \hskip-\@totalleftmargin \hskip\columnwidth}%
 \MakeFramed {\advance\hsize-\width
   \@totalleftmargin\z@ \linewidth\hsize
   \@setminipage}}%
 {\par\unskip\endMakeFramed%
 \at@end@of@kframe}
\makeatother

\definecolor{shadecolor}{rgb}{.97, .97, .97}
\definecolor{messagecolor}{rgb}{0, 0, 0}
\definecolor{warningcolor}{rgb}{1, 0, 1}
\definecolor{errorcolor}{rgb}{1, 0, 0}
\newenvironment{knitrout}{}{} % an empty environment to be redefined in TeX

\usepackage{alltt}
%\usepackage{beamerarticle}

\usepackage[utf8]{inputenc}
\usepackage[T1]{fontenc}
\usepackage[finnish]{babel}
\usepackage{amsthm}
\usepackage{amsmath}
\usepackage{amssymb}
\usepackage{mathtools}
\usepackage[math]{kurier}
\usepackage{eurosym}
\usepackage{caption}
\usepackage{enumerate}
\usepackage{geometry}
\usepackage{hyperref}
\usepackage{graphicx}


%\usepackage[orientation=portrait,size=a4]{beamerposter}
\theoremstyle{remark}
\newtheorem*{esim}{Esimerkki}
\newtheorem*{ratkaisu}{Ratkaisuehdotus}
\IfFileExists{upquote.sty}{\usepackage{upquote}}{}
\begin{document}


	\begin{frame}
		\frametitle{Esimerkki}
			Työntekijän kokonaisveroprosentti on 24. Seuraavana vuonna verotus kevenee yhdellä prosenttiyksiköllä, mutta samalla kuluttajahinnat nousevat 2,4~\%. Jos bruttopalkka pysyy ennallaan, miten muuttui työntekijän reaalinen nettopalkka? Kuinka monta prosenttia 
			palkankorotuksen on oltava, jotta reaalinen nettoansio 
			\begin{enumerate}[(a)]
				\item pysyisi ennallaan
				\item nousisi 3,0~\%?
			\end{enumerate}
		\end{frame}

		\begin{frame}
			\frametitle{Ratkaisuehdotus}
			Olkoon \(a\) alkuperäinen bruttopalkka, jolloin alkuperäinen nettopalkka on \(0,76a\). \pause Jos nimellinen bruttopalkka pysyy ennallaan, seuraavana vuonna käteen jää \(0,77a\). \pause Koska kuluttajahinnat ovat tulleet \(1,024\)-kertaisiksi, uusi reaalinen nettopalkka on \(0,77a/1,024 = 0{,}7519531a\). \pause Tämän osuus alkuperäisestä reaalisesta nettopalkasta on
			\[
				\frac{0{,}7519531a}{0,76a} = \frac{0{,}7519531}{0,76} = 0{,}989412,
			\]
			\pause eli nettopalkka on pienentynyt noin 1{,}06 prosenttia.
		\end{frame}

		\begin{frame}
			Tulkoon sitten uusi palkka \(k\)-kertaiseksi, eli olkoon uusi palkka \(ka\). \pause Tällöin uusi nettopalkka on \(0,77ka\) ja vastaava reaalinen nettopalkka \(0,77ka/1,024\).
		\end{frame}

		\begin{frame}
			\frametitle{(a)}
			\pause
			Jos  reaalinen nettopalkka on sama kuin ennen muutoksia, niin
			\[
				\frac{0,77\cdot ka}{1,024} = 0,76a
			\]
			\pause eli
			\[
				k = \frac{1,024 \cdot 0,76a}{0,77a} = \frac{1,024\cdot0,76}{0,77} = 1{,}010701
			\]
			\pause Siis palkankorotuksen pitäisi olla 
			\[
				1{,}010701 - 1 = 0{,}0107013 \approx 1{,}07~\%.
			\]
		\end{frame}

		\begin{frame}
			\frametitle{(b)}
			\pause
			Jos reaalinen nettopalkka on 3,0~\% suurempi kuin alussa, niin
			\[
				\frac{0,77\cdot ka}{1,024} = 1,03\cdot0,76a
			\]
			\pause eli
			\[
				k = \frac{1,024 \cdot 1,03\cdot 0,76a}{0,77a} = \frac{1,024\cdot1,03\cdot 0,76}{0,77} = 1{,}041022
			\]
			\pause Siis palkankorotuksen pitäisi olla 
			\[
				1{,}041022 - 1 = 0{,}04102234 \approx 4{,}1~\%.
			\]
		\end{frame}

\end{document}
