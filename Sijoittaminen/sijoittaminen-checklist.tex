\documentclass{article}


\usepackage[utf8]{inputenc}
\usepackage[T1]{fontenc}
\usepackage[finnish]{babel}
\usepackage{amsthm}
\usepackage{amsmath}
\usepackage{amssymb}
\usepackage{mathtools}
\usepackage[math]{kurier}
\usepackage{eurosym}
\usepackage{caption}
\usepackage{enumerate}
\usepackage{geometry}
% package for exercises
\usepackage{exsheets}
\DeclareTranslation{finnish}{exsheets-exercise-name}{Tehtävä}
\DeclareTranslation{finnish}{exsheets-question-name}{Kysymys}
\DeclareTranslation{finnish}{exsheets-solution-name}{Ratkaisuehdotus, t.}
%\newtheorem{teht}{Tehtävä}
%\theoremstyle{remark}
%\newtheorem{ratk}{Ratkaisuehdotus}
\makeatletter
\renewcommand\thesection{\hspace*{-1.0em}}
\renewcommand\thesubsection{\hspace*{0.0em}}
\makeatother

\begin{document}

\section{Sijoittaminen - huomioita}\label{sijoittaminen---huomioita}

\subsection{Pankkitalletukset}\label{pankkitalletukset}

\begin{itemize}
\itemsep1pt\parskip0pt\parsep0pt
\item
  Käsitteitä:

  \begin{itemize}
  \itemsep1pt\parskip0pt\parsep0pt
  \item
    Korkokausi
  \item
    Korkokanta
  \item
    Korkoaika
  \item
    Lähdevero
  \end{itemize}
\item
  Korkoaika ilmoitetaan korkokausina (r=kit)
\item
  Korko maksetaan korkokauden päätteeksi
\item
  Koronmaksun yhteydessä peritään lähdevero (30\%)
\item
  (Todellisuudessa lähdevero pyöristetään alaspäin kymmenien senttien
  tarkkuuteen)
\item
  Lähdevero huomioidaan korkokerrointa muodostettaessa
  (nettokorkokerroin)
\end{itemize}

\subsection{Obligaatiot}\label{obligaatiot}

\begin{itemize}
\itemsep1pt\parskip0pt\parsep0pt
\item
  Terminologiaa:

  \begin{itemize}
    \itemsep1pt\parskip0pt\parsep0pt
    \item
      joukkovelkakirjalaina: obligaatio on joukkovelkakirjalainan
      osavelkakirja
    \item
      nimellisarvo: obligaation arvo liikkellelaskuhetkellä (sama, joka
      obligaatiosta maksetaan takaisin laina-ajan päätyttyä)
    \item
      emissiokurssi: obligaation myyntihinta suhteessa nimellisarvoon
      (tämä hinta obligaatiosta maksetaan ostopäivänä)
    \item
      jälkimarkkinahyvitys: kauppaa edeltäneen koronmaksupäivän ja
      kaupantekopäivän väliseltä ajalta maksettava korko
  \end{itemize}
\item
  Käytetään yksinkertaista korkolaskentaa
\item
  Korkotulot lähdeveron alaisia
\item
  Jälkimarkkinahyvitys on 
  \begin{itemize}
    \item 
      myyjälle pääomatuloa
    \item 
      ostajalle verovähennyskelpoinen tulonhankkimiskustannus
  \end{itemize}
\item
  Otetaanko lähdevero huomioon jälkimarkkinahyvitystä laskettaessa?
\end{itemize}

\subsection{Sijoitusvakuutukset}\label{sijoitusvakuutukset}

\begin{itemize}
\itemsep1pt\parskip0pt\parsep0pt
\item
  Tuotot verotetaan pääomatulona (lähdevero ei siis tule kyseeseen)
\item
  Jos edunsaaja on muu kuin sijoittaja, alunperin sijoitetusta summasta
  maksetaan lisäksi lahjavero
\item
  Verot maksetaan vasta sijoitusajan päätyttyä
\end{itemize}

\subsection{Osakkeet}\label{osakkeet}

\subsection{Rahastot}\label{rahastot}

\end{document}