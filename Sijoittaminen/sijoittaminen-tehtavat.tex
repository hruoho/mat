\documentclass{article}\usepackage[]{graphicx}\usepackage[]{color}
%% maxwidth is the original width if it is less than linewidth
%% otherwise use linewidth (to make sure the graphics do not exceed the margin)
\makeatletter
\def\maxwidth{ %
  \ifdim\Gin@nat@width>\linewidth
    \linewidth
  \else
    \Gin@nat@width
  \fi
}
\makeatother

\definecolor{fgcolor}{rgb}{0.345, 0.345, 0.345}
\newcommand{\hlnum}[1]{\textcolor[rgb]{0.686,0.059,0.569}{#1}}%
\newcommand{\hlstr}[1]{\textcolor[rgb]{0.192,0.494,0.8}{#1}}%
\newcommand{\hlcom}[1]{\textcolor[rgb]{0.678,0.584,0.686}{\textit{#1}}}%
\newcommand{\hlopt}[1]{\textcolor[rgb]{0,0,0}{#1}}%
\newcommand{\hlstd}[1]{\textcolor[rgb]{0.345,0.345,0.345}{#1}}%
\newcommand{\hlkwa}[1]{\textcolor[rgb]{0.161,0.373,0.58}{\textbf{#1}}}%
\newcommand{\hlkwb}[1]{\textcolor[rgb]{0.69,0.353,0.396}{#1}}%
\newcommand{\hlkwc}[1]{\textcolor[rgb]{0.333,0.667,0.333}{#1}}%
\newcommand{\hlkwd}[1]{\textcolor[rgb]{0.737,0.353,0.396}{\textbf{#1}}}%

\usepackage{framed}
\makeatletter
\newenvironment{kframe}{%
 \def\at@end@of@kframe{}%
 \ifinner\ifhmode%
  \def\at@end@of@kframe{\end{minipage}}%
  \begin{minipage}{\columnwidth}%
 \fi\fi%
 \def\FrameCommand##1{\hskip\@totalleftmargin \hskip-\fboxsep
 \colorbox{shadecolor}{##1}\hskip-\fboxsep
     % There is no \\@totalrightmargin, so:
     \hskip-\linewidth \hskip-\@totalleftmargin \hskip\columnwidth}%
 \MakeFramed {\advance\hsize-\width
   \@totalleftmargin\z@ \linewidth\hsize
   \@setminipage}}%
 {\par\unskip\endMakeFramed%
 \at@end@of@kframe}
\makeatother

\definecolor{shadecolor}{rgb}{.97, .97, .97}
\definecolor{messagecolor}{rgb}{0, 0, 0}
\definecolor{warningcolor}{rgb}{1, 0, 1}
\definecolor{errorcolor}{rgb}{1, 0, 0}
\newenvironment{knitrout}{}{} % an empty environment to be redefined in TeX

\usepackage{alltt}


\usepackage[utf8]{inputenc}
\usepackage[T1]{fontenc}
\usepackage[finnish]{babel}
\usepackage{amsthm}
\usepackage{amsmath}
\usepackage{amssymb}
\usepackage{mathtools}
\usepackage[math]{kurier}
\usepackage{eurosym}
\usepackage{caption}
\usepackage{enumerate}
\usepackage{geometry}
% package for exercises
\usepackage{exsheets}
\DeclareTranslation{finnish}{exsheets-exercise-name}{Tehtävä}
\DeclareTranslation{finnish}{exsheets-question-name}{Kysymys}
\DeclareTranslation{finnish}{exsheets-solution-name}{Ratkaisuehdotus, t.}
%\newtheorem{teht}{Tehtävä}
%\theoremstyle{remark}
%\newtheorem{ratk}{Ratkaisuehdotus}

%\SetupExSheets{question/print=false}
\SetupExSheets{solution/print=true}
\DeclareTranslation{finnish}{exsheets-solution-name}{Ratkaisuehdotus}
\IfFileExists{upquote.sty}{\usepackage{upquote}}{}
\begin{document}


\section*{Sijoittaminen - luentotehtäviä}

\subsection*{Pankkitalletukset}

Muista korkotuotoista perittävä lähdevero 30 \%.

\begin{question} 
  Sijoitat kahden vuoden määräaikaistilille 3 000 \euro. Tilin vuotuinen korko on 1{,}2 \% ja sijoitusajan lopussa maksettava lisäkorko 5 \%. Mikä oli sijoituksen nettokorkotuotto prosentteina?
\end{question}
\begin{solution}

Sijoitusajan lopussa tilillä on rahaa 
\[
(1+0,7\cdot 0,05)\cdot(1+0,7\cdot 0,012)^2\cdot3000 = 3~157{,}38
\]
euroa. Näin ollen sijoituksen nettotuotto on 157{,}38 euroa eli prosentteina \(157{,}38/3000\approx 5{,}25\)~\%.
\end{solution}

\begin{question} Korkeakorkoiselle tilille talletettiin rahasumma neljäksi vuodeksi. Tilin vuotuinen korkokanta oli 3{,}5 \% ja talletusajan lopussa alkuperäiselle pääomalle maksettiin 7{,}5 \% lisäkorko. Kuinka suuri oli alkuperäinen talletus, jos talletusajan päätyttyä tililtä nostettiin 10~387{,}45 euroa?
\end{question}
\begin{solution}

  Olkoon \(k\) alkuperäinen talletus. Talletusajan lopuksi maksettava lisäkorko on \((0,7\cdot0,075)\cdot k \) euroa, joten tilillä on lopuksi rahaa
  \[
    (0,7\cdot0,075)\cdot k + (1+0,7\cdot0,035)^4\cdot k = 0{,}0525k + 1{,}101661 k = 1{,}154161 k
  \]
euroa. Saadaan yhtälö
 \[
    1{,}154161\cdot k = 10~387{,}45,
  \]
jonka perusteella
\[
  k = \frac{10~387{,}45}{1{,}154161} = 9~000
\]
euroa.
\end{solution}

\subsection*{Vakuutukset}

\begin{question} 
Selvitä kurssimateriaalista, miten sijoitusvakuutusta verotetaan. 
\end{question}
\begin{solution}
Sijoitusvakuutuksesta saaduista tuotoista maksetaan pääomatulovero sijoitusajan päätyttyä. Jos edunsaaja on joku muu kuin sijoittaja itse, maksetaan lisäksi lahjavero sijoitetusta summasta.
\end{solution}

\begin{question} Sijoitat sijoitusvakuutukseen vuoden alussa 1~500~\euro\ ja kahden seuraavan vuoden alussa 1~000~\euro\ vuosittain. Vakuutuksesta pitää maksaa sijoitettavan summan lisäksi hoitokuluja 15 euroa vuodessa. Sijoitusvakuutuksen vuosituotoksi arvioidaan 6~\%. Mikä on sijoituksesi nettotuotto kolmen vuoden kuluttua? 
\end{question}
\begin{solution}


Sijoituksen arvo kolmen vuoden kuluttua on 
\[
  1{,}06^3\cdot1~500 + 1{,}06^2\cdot1~000 + 1{,}06\cdot1~000 = 3~970{,}124
\]
euroa.
Tuotto on siis kulut huomioiden ollut 
\[
  3~970{,}124 - 3~500 45 = 425{,}124
  \]
  euroa. Nettotuotto saadaan kun tuotosta vähennetään pääomatulovero:
\[
  0,7\cdot425{,}124 = 297{,}5868.
\]
Nettotuotto on siis 297{,}59 euroa.
\end{solution}

\begin{question} Isotätisi tallettaa sijoitusvakuutukseen ensimmäisenä vuonna 3~000 euroa ja toisena vuonna 2~000 euroa. Hän maksoi vakuutuksesta sijoitettavan summan lisäksi hoitokuluja 20 euroa vuodessa. Sijoitusvakuutuksen vuosituotto oli ensimmäisenä vuonna 7~\% ja toisena vuonna 4~\%. Edunsaajaksi isotätisi määräsi sinut. Kuinka paljon rahaa sait vakuutuksesta verojen maksamisen jälkeen?
\end{question}
\begin{solution}

Sijoituksen arvo on kahden vuoden kuluttua
\[
  3~000\cdot1{,}07\cdot1{,}04 + 2~000\cdot1{,}04 = 5~418{,}4
\]
euroa. Tuotto on 
\[
  5~418{,}4 - 5~000 - 40 = 378{,}4
\]
ja siten lähdeveron jälkeinen nettotuotto
\[
  0,7\cdot 378{,}4 = 264{,}88
\]
euroa. Itse sijoituksesta joudutaan maksamaan lahjaveroa \(100 + 0,07\cdot(5~000 - 4000) = 800\) euroa, joten kaiken kaikkiaan käteen jää
\[
5~000 - 800 + 264{,}88 = 4~464{,}88
\]
euroa. 
\end{solution}

\subsection*{Obligaatiot}

\begin{question}
  Selvitä kurssimateriaalista, mitä seuraavat käsitteet tarkoittavat:
  \begin{itemize}
    \item joukkovelkakirjalaina
    \item obligaatio
    \item liikkeellelaskupäivä
    \item nimellisarvo
    \item emissiokurssi
    \item jälkimarkkinahyvitys
  \end{itemize}
  Selvitä myös, miten obligaatioista saatua tuottoa verotetaan.
\end{question}

\begin{question} Kaverisi osti nimellisarvoltaan 2~000 eurolla obligaatioita 95~\% emissiokurssilla 73 päivää lainan liikkeellelaskun jälkeen. Obligaatioiden korkokanta oli 4{,}8~\% p.a. ja laina-aika kymmenen vuotta. Korkopäivien laskemiseen käytettiin tapaa todelliset/365.
  \begin{enumerate}
    \item Kuinka paljon kaverisi maksoi obligaatioistaan yhteensä?
    \item Kuinka paljon kaverisi sai sijoituksestaan nettotuottoa, jos hän myi obligaatiot viiden vuoden kuluttua liikkeellelaskupäivästä hintaan 2 200 euroa?
  \end{enumerate} 
\end{question}
\begin{solution}
  \begin{enumerate}[(a)]
    \item Emissiokurssin mukainen hinta oli \(0,95\cdot2000 = 1~900\) euroa. Jälkimarkkinahyvitys 73:lta päivältä oli \(2000\cdot0,048\cdot\frac{73}{365} = 19{,}2\) euroa. Yhteensä maksettavaa kertyi siis \(1~900 + 19{,}2 = 1~919{,}2\) euroa.
    \item 
      Vuosittainen korkotuotto oli \(0,048\cdot2000 = 96\) euroa. Viideltä vuodelta korkotuloja kertyi siis 480 euroa, joista lähdeveron jälkeen jäi käteen \(336\) euroa. Myyntivoittoa (pääomatuloa) hän sai \(2200 - 1~900 - 19{,}2 = 280{,}8\), joista pääomatuloveron jälkeen jäi \(196{,}56\) euroa. Kaiken kaikkiaan nettotuloja kertyi \(532{,}56\) euroa. 
  \end{enumerate}
\end{solution}

\begin{question} Valtion liikkeellelaskeman obligaatiolainan vuosikorko oli 5 \%. Sijoittaja osti kolme obligaatiota 108 päivää liikkeellelaskun jälkeen, kun emissiokurssi oli 102 \%. Mikä oli yhden obligaation nimellisarvo, jos sijoittaja maksoi obligaatioista yhteensä 4 657{,}50 euroa? Käytä korkopäivien laskemiseen tapaa todelliset/360.
\end{question}
\begin{solution}

  Olkoon \(x\) obligaatioiden yhteinen nimellisarvo. Sijoittaja maksoi emissiokurssin mukaisen hinnan lisäksi 108 päivältä kertyneen koron. Emissiokurssin mukainen hinta oli \(1,02x\) euroa ja koron suuruus
  \[
  x\cdot0,05\cdot\frac{108}{360} = 0{,}015x
  \]
euroa. Nyt tiedetään, että
\[
  1,02x + 0{,}015x = 4~657{,}5
\]
eli
\[
  1{,}035x = 4~657{,}5.
\]
Tästä saadaan
\[
  x = \frac{4~657{,}5}{1{,}035} = 4~500,
\]
eli kolmen obligaation yhteinen nimellisarvo oli \(4~500\) euroa. Näin ollen yhden obligaation nimellisarvo oli \(1~500\) euroa.
\end{solution}

\subsection*{Osakkeet}

\begin{enumerate}
  \item Selvitä kurssimateriaalista, mitä tarkoittavat käsitteet osake ja osinko.
  \item Selviä kurssimateriaalista, miten pörssiin listattujen yhtiöiden osakkeiden tuottoa verotetaan.
\end{enumerate}
 
\begin{question} 
    Kaverisi osti 600 kappaletta erään pörssiin listatun yrityksen osakkeita 9{,}45 euron kappalehintaan. Keväällä yritys jakoi jokaiselta osakkeelta osinkoa 1{,}15~\euro. Vuoden lopulla kaverisi myi osakkeet 10{,}9 euron kappalehintaan. Sekä oston että myynnin yhteydessä perittiin välityspalkkiota 1~\% kauppahinnasta. Mikä oli kaverisi sijoituksen nettotuotto?
\end{question}
\begin{solution}
    Osakkeet maksoivat 5~670 euroa ja välityspalkkiota maksettiin 56{,}7 euroa. Osinkotuloja saatiin 690 euroa. Osakkeiden myynnistä saatiin 6~540 euroa ja välityspalkkiota maksettiin 65{,}4 euroa. 
    Osinkotuloista 15~\% on verotonta ja 85~\% veronalaista pääomatuloa ja myyntivoitto on kokonaisuudessaan veronalaista pääomatuloa. Pääomatulosta voidaan vähentää tulonhankkimiskulut \(5~670 + 56{,}7 +65{,}4 = 5~792{,}1\) euroa. Verotettava pääomatulo on näin ollen
    \[
        0,85\cdot690 + 6~540 - 5~792{,}1 = 1~334{,}4
    \]
    euroa. Nettotuloiksi jää \(0,7\cdot1~334{,}4 = 934{,}08\) euroa, joten kaiken kaikkiaan nettotuotto oli
    \[
        0,15\cdot690 + 934{,}08 = 1~037{,}58
    \]
    euroa. 
\end{solution}

\begin{question}
    Sijoittaja osti 90 kappaletta erään yhtiön osakkeita kurssilla 35{,}86~\euro. Yhdeksän kuukauden kuluttua hän myi osakkeet kurssilla 37{,}23~\euro. Välityspalkkio oli sekä ostettaessa että myytäessä $0{,}25\ \%  + 7{,}50 \euro$. Mikä oli sijoituksen nettotuottoa vastaava vuosittainen korkokanta?
\end{question}
\begin{solution}
    Pääomatuloista \(90\cdot37{,}23 = 3~350{,}7\)~\euro\ voidaan vähentää tulonhankkimiskustannuksina ostohinta \(90\cdot35{,}86 = 3~227{,}4\) euroa ja välityspalkkiot \(0,0025\cdot3~227{,}4 + 7,5 + 0,0025\cdot3~350{,}7 + 7,5 = 31{,}44525\) euroa. Verotettavaksi tuloksi jää siis 
    \[
        3~350{,}7 - 3~227{,}4 - 31{,}44525 = 91{,}85475,
    \]
    joten nettotuotto on \(0,7\cdot91{,}85475 = 64{,}29832\) euroa. Sijoitus kasvoi siis yhdeksässä kuukaudessa
    \[
        \frac{64{,}29832}{3~227{,}4} = 0{,}01992264 \approx 1{,}99~\%,
    \]
    joten vuosikorko olisi 2{,}66~\%.
\end{solution}

\subsection*{Sijoitusrahastot}

\begin{question} Selvitä kurssimateriaalista, miten sijoitusrahastojen tuottoa verotetaan.

\end{question}

\begin{question} 
    Kaverisi osti korkorahaston osuuksia 1 200 eurolla. Hän myi osuudet vuoden kuluttua, jolloin osuuksien arvo oli 1 350 euroa. Kauppojen yhteydessä perityt välityspalkkiot olivat sekä oston että myynnin tapauksessa 0{,}95 \% kauppahinnasta. Mikä oli sijoituksen nettotuottoa vastaava vuosittainen korkokanta?
\end{question}
\begin{solution}
    Nettotuotto on 
    \[
        0,7\cdot(1350 - 1200 - 0,0095\cdot(1350+1200)) = 88{,}0425
    \]
    Sijoitus kasvoi siis vuodessa korkoa
    \[
        \frac{88{,}0425}{1200} = 0{,}07336875 \approx 7{,}34~\%
    \]
\end{solution}

\begin{question} 
    Sijoitit rahastoon 95 \euro\ ja sait 8{,}5 rahasto-osuutta. Kahden kuukauden kuluttua rahasto-osuuden arvo oli noussut 3{,}2 \%. Kuinka monta rahasto-osuutta sait, jos sijoitit saman summan kuin edellisellä kerralla?
\end{question}
\begin{solution}
    Rahasto-osuuden arvo oli \(95/8,5 = 11{,}17647\). Nousun jälkeen arvo oli 11{,}53412, joten osuuksi saatiin
    \[
        \frac{95}{11{,}53412} = 8{,}236434 \approx 8{,}24
    \]
    kpl.
\end{solution}

\subsection*{Sekalaisia tehtäviä}

\begin{question} 
    Sijoitat loppuvuoden aikana erääseen rahastoon joka kuukausi saman summan syyskuusta alkaen. Rahasto-osuuden arvo euroina on alla olevassa taulukossa. Oletko voitolla vai tappiolla sijoituksissasi joulukuun sijoituksen jälkeen? Kuinka monta prosenttia voitto tai tappio on kokonaissijoituksestasi?
% latex table generated in R 3.0.2 by xtable 1.7-4 package
% Sat Apr 25 13:44:54 2015
\begin{table}[ht]
\centering
\begin{tabular}{rrrr}
  \hline
syyskuu & lokakuu & marraskuu & joulukuu \\ 
  \hline
2,70 & 2,95 & 2,80 & 2,65 \\ 
   \hline
\end{tabular}
\end{table}

\end{question}
\begin{solution}
    Taulukoidaan:
% latex table generated in R 3.0.2 by xtable 1.7-4 package
% Sat Apr 25 13:44:54 2015
\begin{table}[ht]
\centering
\begin{tabular}{rlrl}
  \hline
 & Kuukausi & Arvo & Uusia osuuksia \\ 
  \hline
1 & syyskuu & 2,70 & x / 2.7 \\ 
  2 & lokakuu & 2,95 & x / 2.95 \\ 
  3 & marraskuu & 2,80 & x / 2.8 \\ 
  4 & joulukuu & 2,65 & x / 2.65 \\ 
   \hline
\end{tabular}
\end{table}

\noindent
Joulukuun lopussa rahasto-osuuksia on 
    \[  
        \frac{x}{2,7} + \frac{x}{2,95} + \frac{x}{2,80} + \frac{x}{2,65} = 1{,}443855x
    \]
    kappaletta, joten niiden arvo on 
    \[
        2,65\cdot1{,}443855x = 3{,}826215x
    \] euroa. Osuuksien hankintaan on puolestaan kulunut \(4x\) euroa, joten tappiolla ollaan. Joulukuun rahasto-osuuksien arvo on 
    \[
        \frac{3{,}826215x}{4x} = 0{,}9565538 \approx 95{,}66~\%
    \]
    ostohinnasta, joten tappiolla ollaan noin 4{,}34~\%. 

\end{solution}

\begin{question} 
    Kaverisi tallettaa 16.4.2014 määräaikaistilille 2 400 euroa. Tilin eräpäiväksi asetetaan 20.3.2015. Tilin kiinteä korko on 2{,}95 \% p.a. ja se lasketaan korkotavan todelliset/360 mukaan. Korko liitetään pääomaan vuoden vaihteessa ja talletuksen eräpäivänä. Kuinka paljon kaverisi voi nostaa tililtään eräpäivänä?
\end{question}
\begin{solution}
    Vuodelta 2014 korkopäiviä kertyy 261 kpl ja vuodelta 2015 79 kpl. Vuoden 2014 korko on
    \[
        r = 2400\cdot 0,0295\cdot \frac{261}{360} = 51{,}33
    \]
    euroa, joten vuoden 2015 alussa tilillä on lähdevero huomioiden 
    \[
        k = 2400 + 0,751{,}33 = 2~435{,}931
    \]
    euroa. Näin ollen vuodelta 2015 saadaan
    \[
        r = 2~435{,}931\cdot0,0295\cdot\frac{79}{360} = 15{,}76927
    \]
    euroa korkoa, josta lähdeveron jälkeen jää 11{,}03849\ \euro. Niinpä tililtä on 20.3.2015 nostettavissa 2~446{,}97 euroa.
\end{solution}

\begin{question} 
    Ostat 50 kappaletta erään yrityksen osakkeita hintaan 21{,}75 \euro/kpl. Välityspalkkio on 0{,}5~\% kauppahinnasta, kuitenkin vähintään 8 \euro. Kuinka paljon maksat yhdestä osakkeesta, jos välityspalkkio otetaan huomioon?
\end{question}
\begin{solution}
    Osakkeiden ostohinta on \(50\cdot21,75 = 1~087{,}5\) euroa, joten 0,5~\% välityspalkkioksi saadaan \(5{,}4375\) euroa. Välityspalkkiota joudutaan siis maksamaan 8 euroa. Lopullinen hinta on \(1~095{,}5\)~\euro, eli osaketta kohden \(21{,}91\) euroa.
\end{solution}

\begin{question} 
    Joukkovelkakirjalainan korkokanta on 5{,}5 \% p.a. ja laina-aika neljä vuotta. Ostat tämän lainan obligaatioita 144 päivää liikkeellelaskun jälkeen, kun emissiokurssi on 98{,}5 \%. Ostamiesi obligaatioiden nimellisarvo on yhteensä 2 500 euroa. Jos olisit tallettanut obligaatioista maksamasi summan pankkitilille, mikä olisi sen vuotuisen korkokannan pitänyt olla, jotta sijoitusten nettotuotot olisivat yhtä suuria? Korkotapana 30/360.
\end{question}
\begin{solution}
    Obligaatiosta maksettiin emissiokurssin mukainen hinta ja korko 144 päivältä, eli yhteensä 
    \[
        0,985\cdot2500 + 2500\cdot0,055\cdot\frac{144}{360} = 2~517{,}5
    \]
    euroa. Laina ajan päätyttyä saadaan nimellishinnan lisäksi kolmelta vuodelta kertyneet korot, eli yhteensä
    \[
       2500 + 3\cdot 2500 \cdot 0,055 = 2~912{,}5
    \]
    euroa. Voittoa (ennen veroja) saatiin siis \(395\) euroa. Jotta tililtä, jonka korkokanta on \(i\) saataisiin sama korkotuotto, täytyisi olla
    \[
        2~517{,}5\cdot (1+i)^4 - 2~517{,}5 = 395
    \]
    eli
    \[
        i = 0{,}03710831.
    \]
    Tilin korkokannan olisi siis oltava \(3{,}71\)~\%.
\end{solution}

\begin{question} 
    Talletat sijoitusvakuutukseen kolmena peräkkäisenä vuonna jokaisen vuoden alussa 2~000 euroa. Sijoitusvakuutuksen vuotuinen tuotto on ensimmäisenä vuonna 3~\%, toisena vuonna 2~\% ja kolmantena vuonna 2{,}5~\%. Maksat sijoittamiesi rahasummien lisäksi vakuutuksesta hoitomaksua 15~\euro\ vuodessa. 
    \begin{enumerate}[(a)]
        \item Mikä on sijoituksesi nettotuotto sijoitusajan päättyessä kolmannen vuoden lopussa? 
        \item Jos olisit tallettanut sijoittamasi summat samalla tavalla vuosittain pankkitilille, jonka korkokanta on 2{,}5 \% p.a., olisiko nettotuotto ollut suurempi vai pienempi kuin sijoitusvakuutuksessa?
    \end{enumerate}
\end{question}
\begin{solution}

    \begin{enumerate}[(a)]
        \item Sijoituksen arvo korkoineen sijoitusajan lopussa on 6~294{,}73, joten korkoa on saatu 294{,}73 euroa. Pääomatuloveroa maksetaan siis \(294{,}73 - 3\cdot15 = 249{,}73\) eurosta, joten korosta jää käteen 
        \[
           0,7\cdot249{,}73 = 174{,}811 \approx 174{,}81
        \]
        euroa. Tämä on sijoituksen nettotuotto.
        \item Pankkitilillä olisi sijoituksen päätyttyä
        \[
            2000\cdot(1+0,7\cdot0,025)^3 + 2000\cdot(1+0,7\cdot0,025)^2+ 2000\cdot(1+0,7\cdot0,025) = 6~212{,}461
        \]
        euroa, joten korkoa olisi saatu 212{,}46 euroa. Pankkitililtä saatu nettotuotto olisi siis tässä tapauksessa ollut suurempi.
    \end{enumerate}
\end{solution}



\end{document}
