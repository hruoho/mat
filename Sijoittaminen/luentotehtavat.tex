\documentclass{article}\usepackage[]{graphicx}\usepackage[]{color}
%% maxwidth is the original width if it is less than linewidth
%% otherwise use linewidth (to make sure the graphics do not exceed the margin)
\makeatletter
\def\maxwidth{ %
  \ifdim\Gin@nat@width>\linewidth
    \linewidth
  \else
    \Gin@nat@width
  \fi
}
\makeatother

\definecolor{fgcolor}{rgb}{0.345, 0.345, 0.345}
\newcommand{\hlnum}[1]{\textcolor[rgb]{0.686,0.059,0.569}{#1}}%
\newcommand{\hlstr}[1]{\textcolor[rgb]{0.192,0.494,0.8}{#1}}%
\newcommand{\hlcom}[1]{\textcolor[rgb]{0.678,0.584,0.686}{\textit{#1}}}%
\newcommand{\hlopt}[1]{\textcolor[rgb]{0,0,0}{#1}}%
\newcommand{\hlstd}[1]{\textcolor[rgb]{0.345,0.345,0.345}{#1}}%
\newcommand{\hlkwa}[1]{\textcolor[rgb]{0.161,0.373,0.58}{\textbf{#1}}}%
\newcommand{\hlkwb}[1]{\textcolor[rgb]{0.69,0.353,0.396}{#1}}%
\newcommand{\hlkwc}[1]{\textcolor[rgb]{0.333,0.667,0.333}{#1}}%
\newcommand{\hlkwd}[1]{\textcolor[rgb]{0.737,0.353,0.396}{\textbf{#1}}}%

\usepackage{framed}
\makeatletter
\newenvironment{kframe}{%
 \def\at@end@of@kframe{}%
 \ifinner\ifhmode%
  \def\at@end@of@kframe{\end{minipage}}%
  \begin{minipage}{\columnwidth}%
 \fi\fi%
 \def\FrameCommand##1{\hskip\@totalleftmargin \hskip-\fboxsep
 \colorbox{shadecolor}{##1}\hskip-\fboxsep
     % There is no \\@totalrightmargin, so:
     \hskip-\linewidth \hskip-\@totalleftmargin \hskip\columnwidth}%
 \MakeFramed {\advance\hsize-\width
   \@totalleftmargin\z@ \linewidth\hsize
   \@setminipage}}%
 {\par\unskip\endMakeFramed%
 \at@end@of@kframe}
\makeatother

\definecolor{shadecolor}{rgb}{.97, .97, .97}
\definecolor{messagecolor}{rgb}{0, 0, 0}
\definecolor{warningcolor}{rgb}{1, 0, 1}
\definecolor{errorcolor}{rgb}{1, 0, 0}
\newenvironment{knitrout}{}{} % an empty environment to be redefined in TeX

\usepackage{alltt}

\usepackage[utf8]{inputenc}
\usepackage[T1]{fontenc}
\usepackage[finnish]{babel}
\usepackage{amsthm}
\usepackage{amsmath}
\usepackage{amssymb}
\usepackage{mathtools}
\usepackage[math]{kurier}
\usepackage{eurosym}

\newtheorem{teht}{Tehtävä}
\theoremstyle{remark}
\newtheorem*{ratk}{Ratkaisuehdotus}
\usepackage{environ}
\NewEnviron{killcontents}{}
\let\ratk\killcontents
\let\endratk\endkillcontents
\IfFileExists{upquote.sty}{\usepackage{upquote}}{}
\begin{document}

\section*{Luentotehtäviä}

\subsection*{Pankkitalletukset}

Muista korkotuotoista perittävä lähdevero 30 \%.

\begin{teht} 
  Sijoitat kahden vuoden määräaikaistilille 3 000 \euro. Tilin vuotuinen korko on 1{,}2 \% ja sijoitusajan lopussa maksettava lisäkorko 5 \%. Mikä oli sijoituksen nettokorkotuotto prosentteina?
\end{teht}
\begin{ratk}

Sijoitusajan lopussa tilillä on rahaa 
\[
(1+0,7\cdot 0,05)\cdot(1+0,7\cdot 0,12)^2\cdot3000 = 3~157,38
\]
euroa. Näin ollen sijoituksen nettotuotto on 157,38 euroa eli prosentteina \(157,38/3000\approx 5,25\)~\%.
\end{ratk}

\begin{teht} Korkeakorkoiselle tilille talletettiin rahasumma neljäksi vuodeksi. Tilin vuotuinen korkokanta oli 3{,}5 \% ja talletusajan lopussa alkuperäiselle pääomalle maksettiin 7{,}5 \% lisäkorko. Kuinka suuri oli alkuperäinen talletus, jos talletusajan päätyttyä tililtä nostettiin 10~387{,}45 euroa?
\end{teht}
\begin{ratk}

  Olkoon \(k\) alkuperäinen talletus. Huomaa, että lisäkorko maksetaan alkuperäiselle pääomalle. Näin ollen tilillä on lopuksi rahaa
  \[
    (0,7\cdot0,075)\cdot k + (1+0,7\cdot0,035)^4\cdot k = (0,7\cdot0,075 + (1+0,7\cdot0,035)^4)\cdot k
  \]
euroa. Saadaan yhtälö
 \[
    1,154161\cdot k = 10~387,45
  \]
jonka perusteella
\[
  k = \frac{10~387,45}{1,154161} = 9~000
\]
euroa.
\end{ratk}

\subsection*{Vakuutukset}

\begin{teht} 
Selvitä kurssimateriaalista, miten sijoitusvakuutusta verotetaan. 

\end{teht}\begin{ratk}\end{ratk}

\begin{teht} Sijoitat sijoitusvakuutukseen vuoden alussa 1 500 \euro \ ja kahden seuraavan vuoden alussa 1 000 \euro \ vuosittain. Vakuutuksesta pitää maksaa sijoitettavan summan lisäksi hoitokuluja 15 euroa vuodessa. Sijoitusvakuutuksen vuosituotoksi arvioidaan 6 \%. Mikä on sijoituksesi nettotuotto kolmen vuoden kuluttua? 

\end{teht}\begin{ratk}\end{ratk}

\begin{teht} Isotätisi tallettaa sijoitusvakuutukseen ensimmäisenä vuonna 3 000 euroa ja toisena vuonna 2 000 euroa. Hän maksoi vakuutuksesta sijoitettavan summan lisäksi hoitokuluja 20 euroa vuodessa. Sijoitusvakuutuksen vuosituotto oli ensimmäisenä vuonna 7 \% ja toisena vuonna 4 \%. Edunsaajaksi isotätisi määräsi sinut. Kuinka paljon rahaa sait vakuutuksesta verojen maksamisen jälkeen?
\end{teht}

\subsection*{Obligaatiot}

\begin{teht} Kaverisi osti nimellisarvoltaan 2 000 eurolla obligaatioita 95 \% emissiokurssilla 73 päivää lainan liikkeellelaskun jälkeen. Obligaatioiden korkokanta oli 4{,}8 \% p.a. ja laina-aika kymmenen vuotta. Korkopäivien laskemiseen käytettiin tapaa todelliset/365.
\begin{enumerate}
  \item Kuinka paljon kaverisi maksoi obligaatioistaan yhteensä?
  \item Kuinka paljon kaverisi sai sijoituksestaan nettotuottoa, jos hän myi obligaatiot viiden vuoden kuluttua liikkeellelaskupäivästä hintaan 2 200 euroa?
\end{enumerate}

\end{teht}\begin{ratk}\end{ratk}

\begin{teht} Valtion liikkeellelaskeman obligaatiolainan vuosikorko oli 5 \%. Sijoittaja osti kolme obligaatiota 108 päivää liikkeellelaskun jälkeen, kun emissiokurssi oli 102 \%. Mikä oli yhden obligaation nimellisarvo, jos sijoittaja maksoi obligaatioista yhteensä 4 657{,}50 euroa? Käytä korkopäivien laskemiseen tapaa todelliset/360.
\end{teht}

\subsection*{Osakkeet}

\begin{enumerate}
  \item Selvitä kurssimateriaalista, mitä tarkoittavat käsitteet osake ja osinko.
  \item Selviä kurssimateriaalista, miten pörssiin listattujen yhtiöiden osakkeiden tuottoa verotetaan.
\end{enumerate}
 
\begin{teht} Kaverisi osti 600 kappaletta erään pörssiin listatun yrityksen osakkeita 9{,}45 euron kappalehintaan. Keväällä yritys jakoi jokaiselta osakkeelta osinkoa 1{,}15 \euro. Vuoden lopulla kaverisi myi osakkeet 10{,}90 euron kappalehintaan. Sekä oston että myynnin yhteydessä perittiin välityspalkkiota 1{,}0 \% kauppahinnasta. Mikä oli kaverisi sijoituksen nettotuotto?
\end{teht}\begin{ratk}\end{ratk}

\begin{teht} Sijoittaja osti 90 kappaletta erään yhtiön osakkeita kurssilla 35{,}86 \euro. Yhdeksän kuukauden kuluttua hän myi osakkeet kurssilla 37{,}23 \euro. Välityspalkkio oli sekä ostettaessa että myytäessä $0{,}25\ \%  + 7{,}50 \euro$. Mikä oli sijoituksen nettotuottoa vastaava vuosittainen korkokanta?
\end{teht}

\subsection*{Sijoitusrahastot}

\begin{teht} Selviä kurssimateriaalista, miten sijoitusrahastojen tuottoa verotetaan.

\end{teht}

\begin{teht} Kaverisi osti korkorahaston osuuksia 1 200 eurolla. Hän myi osuudet vuoden kuluttua, jolloin osuuksien arvo oli 1 350 euroa. Kauppojen yhteydessä perityt välityspalkkiot olivat sekä oston että myynnin tapauksessa 0{,}95 \% kauppahinnasta. Mikä oli sijoituksen nettotuottoa vastaava vuosittainen korkokanta?

\end{teht}\begin{ratk}\end{ratk}

\begin{teht} Sijoitit rahastoon 95 \euro \  ja sait 8{,}5 rahasto-osuutta. Kahden kuukauden kuluttua rahasto-osuuden arvo oli noussut 3{,}2 \%. Kuinka monta rahasto-osuutta sait, jos sijoitit saman summan kuin edellisellä kerralla?

\end{teht}

\subsection*{Sekalaisia tehtäviä}

\begin{teht} Sijoitat loppuvuoden aikana erääseen rahastoon joka kuukausi saman summan syyskuusta alkaen. Rahasto-osuuden arvo euroina on alla olevassa taulukossa. Oletko voitolla vai tappiolla sijoituksissasi joulukuun sijoituksen jälkeen? Kuinka monta prosenttia voitto tai tappio on kokonaissijoituksestasi?
\end{teht}\begin{ratk}\end{ratk}

\begin{teht} Kaverisi tallettaa 16.4.2014 määräaikaistilille 2 400 euroa. Tilin eräpäiväksi asetetaan 20.3.2015. Tilin kiinteä korko on 2{,}95 \% p.a. ja se lasketaan korkotavan todelliset/360 mukaan. Korko liitetään pääomaan vuoden vaihteessa ja talletuksen eräpäivänä. Kuinka paljon kaverisi voi nostaa tililtään eräpäivänä?

\end{teht}\begin{ratk}\end{ratk}

\begin{teht} Ostat 50 kappaletta erään yrityksen osakkeita hintaan 21{,}75 \euro/kpl. Välityspalkkio on 0{,}5~\% kauppahinnasta, kuitenkin vähintään 8 \euro. Kuinka paljon maksat yhdestä osakkeesta, jos välityspalkkio otetaan huomioon?

\end{teht}

\begin{ratk}\end{ratk}\begin{teht} Joukkovelkakirjalainan korkokanta on 5{,}5 \% p.a. ja laina-aika kolme vuotta. Ostat tämän lainan obligaatioita 144 päivää liikkeellelaskun jälkeen, kun emissiokurssi on 98{,}5 \%. Ostamiesi obligaatioiden nimellisarvo on yhteensä 2 500 euroa. Jos olisit tallettanut obligaatioista maksamasi summan pankkitilille, mikä olisi sen vuotuisen korkokannan pitänyt olla, jotta sijoitusten nettotuotot olisivat yhtä suuria? Korkotapana 30/360.

\end{teht}\begin{ratk}\end{ratk}

\begin{teht} Talletat sijoitusvakuutukseen kolmena peräkkäisenä vuonna jokaisen vuoden alussa 2 000 euroa. Sijoitusvakuutuksen vuotuinen tuotto on ensimmäisenä vuonna 3 \%, toisena vuonna 2 \% ja kolmantena vuonna 2{,}5 \%. Maksat sijoittamiesi rahasummien lisäksi vakuutuksesta hoitomaksua 15 \euro \ vuodessa. 
\begin{enumerate}
  \item Mikä on sijoituksesi nettotuotto sijoitusajan päättyessä kolmannen vuoden lopussa? 
  \item Jos olisit tallettanut sijoittamasi summat samalla tavalla vuosittain pankkitilille, jonka korkokanta on 2{,}5 \% p.a., olisiko nettotuotto ollut suurempi vai pienempi kuin sijoitusvakuutuksessa?
\end{enumerate}
\end{teht}

\begin{ratk}\end{ratk}



\end{document}
