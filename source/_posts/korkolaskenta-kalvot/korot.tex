\documentclass{beamer}
\input{../../_texPartials/preamble}
\usecolortheme{whale}
\setbeamercolor{block title}{use=frametitle, bg=frametitle.bg, fg=frametitle.fg}
\setbeamercolor{block body}{use=frametitle, bg=frametitle.bg!40!white, fg=black}
\usepackage{eurosym}
\usepackage{hyperref}
\usepackage{graphicx}
\newcommand{\pblock}{\\ \vspace{0.5cm}\pause}
\newcommand{\socrativeOhje}{
\begin{itemize}
\item Surffaa osoitteeseen \url{m.socrative.com} (tai \url{socrative.com})
\item Siirry huoneeseen nimeltä \url{hruoho}
\end{itemize}
}

\newcommand{\taukoKysymys}{
\socrativeOhje
	\begin{block}{Kysymys}
	Mielestäni tähän väliin täydellinen tauko on
	\begin{enumerate}[(A)]
		\item 10 min
		\item 20 min
		\item 30 min
		\item Vähemmän 
		\item Enemmän
	\end{enumerate}	
	\end{block}
}
\title{Talousmatematiikkaa: korkolaskenta}
\begin{document}
\begin{frame}
\maketitle
\end{frame}
\begin{frame}
    \frametitle{Korkolaskenta}
    Tarkastellaan seuraavassa pääasiassa talletusten kasvamia korkoja
    \pblock
    Korkolaskenta jaetaan \emph{yksinkertaiseen korkolaskentaan} ja \emph{koronkorkolaskuihin}
    \pblock
    Keskeisiä käsitteitä mm.
    \begin{itemize}
        \item korkoaika (aika, jonka talletus kasvaa korkoa),
        \item korkokausi (koron maksuväli),
        \item korkokanta (korkoprosentti),
        \item lähdevero (30\%)
    \end{itemize}
\end{frame}

\begin{frame}
    \frametitle{Yksinkertainen korkolasku}
    Kun korkoaika on lyhyempi kuin korkokausi, puhutaan yksinkertaisesta korkolaskusta.
    \pause
    \begin{block}{}
        Korko \(r\) riippuu alkuperäisestä pääomasta \(k\), korkokannasta \(i\) ja korkoajasta \(t\):\pause
        \[
        r = kit
        \]\pause
        Kasvanut pääoma \(K\) saadaan lisäämällä korko alkuperäiseen pääomaan:\pause
        \[
        K = k + r = k + kit
        \]
    \end{block}
    Huomaa, että korkoaika on ilmaistava samassa yksikössä, kuin korkokausi.
\end{frame}

\begin{frame}
    \frametitle{Korkolaskenta}
    \begin{esim}
        Tilin korkokanta on 2,2 \% p.a. ja tilille talletetaan 2 500 euroa.
        Kuinka paljon talletukselle maksetaan kahdeksan kuukauden ajalta korkoa? Huomioi korosta perittävä 30 \% lähdevero.
    \end{esim}
    \pause
    Tässä esimerkissä
    \begin{itemize}
        \item korkoaika: 8 kk eli 8/12 vuotta
        \item korkokausi: 1 vuosi
        \item korkokanta: 2,2 \%
        \item lähdevero: 30 \%
    \end{itemize}
\end{frame}

\begin{frame}
    \pause
    \begin{ratkaisu}
        Kun lähdevero otetaan huomioon, korko on
         \[
            r = kit = 2500\cdot 0,70\cdot 0,022 \cdot \frac{8}{12}\approx 25,67
         \]
         euroa.
    \end{ratkaisu}
\end{frame}

\begin{frame}
    \frametitle{Korkolaskenta}
    \pause Usein käytettyjä korkokausia lyhenteineen:
    \begin{itemize}
        \item kerran vuodessa: p.a. (\emph{per anno}) \pause
        \item kerran puolessa vuodessa: p.s. (\emph{per season}) \pause
        \item kerran neljännesvuodessa: p.q. (\emph{per quarter}) \pause
        \item kerran kuussa: per kk \pause
    \end{itemize}
    Huomaa, että kertynyt korko liitetään pääomaan aina korkokauden \emph{lopussa}
\end{frame}


\begin{frame}
    \frametitle{Yksinkertainen korkolaskenta: nettokorko}
    \pause
    \begin{itemize}
        \item \emph{Nettokorko} tarkoittaa korkoa, joka jää jäljelle lähdeveron maksamisen jälkeen
        \item \emph{Nettokorkokanta} tarkoittaa korkokantaa, jonka avulla nettokorko voidaan laskea suoraan
    \end{itemize}
    \pause
    \begin{esim}
        Talletetaan 9000 euroa tilille, jonka korkokanta on 0,75 \% p.a.
        \pause Laske koron ja nettokoron suuruus, kun talletusaika on kolme kuukautta.
    \end{esim}
    \pause
\end{frame}

\begin{frame}
    \begin{ratkaisu}
        Nyt korkokanta on \pause $i = 0{,}75 \ \%  = 0{,}0075$ ja nettokorkokanta \pause $0{,}70i = 0{,}70 \cdot 0{,}0075 = 0,00525$.
        Talletusaika, ts. korkoaika, on 3/12 vuotta. Siten koron suuruus on
        \[
            r = kit = 9000\cdot 0,0075\cdot \frac{3}{12}\approx 16,88
        \]
        euroa ja nettokoron
        \[
            9000\cdot 0,00525\cdot\frac{3}{12}\approx 11,81
        \]
        euroa.
    \end{ratkaisu}
\end{frame}

\begin{frame}
    \begin{esim}
        Mikä pääoma tuottaa neljässä kuukaudessa nettokorkoa 150 euroa, jos korkokanta on 3{,}6 \% p.a.? Muista, että lähdevero on 30 \%.
    \end{esim}\pause
    \begin{ratkaisu}
        Olkoon \(k\) tuntematon pääoma. \pause Nettokorkokanta on \(0,70\cdot0,036 = 0,0252\).
        \pause Korkokausi on yksi vuosi ja korkoaika 4/12 vuotta.
        \pause Kaavan mukaan on oltava
        \[
            150 = k\cdot 0,0252\cdot \frac{4}{12} = k\cdot 0,0084,
        \]
        \pause
        joten
        \[
            k = \frac{150}{0,0084}\approx 17857,14
        \]
        euroa.
    \end{ratkaisu}
\end{frame}

\begin{frame}
    \frametitle{Yksinkertainen korkolaskenta: lähdeveron pyöristyssääntö}
    \pause
    %\begin{esim}
     %   Talletat huhtikuun alussa 100 000 euroa tilille, jonka korkokanta on 2{,}6~\%~p.a.
      %  Kuinka paljon voit nostaa tililtäsi 100 päivän kuluttua? Huomioi 30 \% lähdevero ja sen pyöristyssääntö.
    %\end{esim}
    \pause
    Lähdeveroon liittyy erikoinen \emph{pyöristyssääntö}:
    lähdevero lasketaan jokaisesta maksetusta korkoerästä täysin kymmenin sentein siten,
    että yli menevät sentit jätetään huomioimatta.
    \pblock
    Pyöristyssääntö jätetään usein huomioimatta tehtävien ratkaisuissa.
\end{frame}

\begin{frame}
    \frametitle{Yksinkertainen korkolaskenta: korkopäivät}
    \pause
    Koron laskeminen voi vaatia \emph{korkopäivien lukumäärän} selvittämisen.
    \pause
    \begin{esim}
        800 euron talletukselle maksetaan vuotuista korkoa 9 \%. Kuinka paljon korkoa kertyy ajalta 2.4.--16.6.2014?
    \end{esim}
    \pause
    Korkoaikaa selvitettäessä talletuspäivää ei lasketa mukaan, nostopäivä lasketaan.
\end{frame}

\begin{frame}
    \begin{ratkaisu}
        Nettokorkokanta on \(0,70\cdot0,09 = 0,063\). Lasketaan korkopäivien lukumäärät:
        \begin{itemize}
            \item Huhtikuussa on 30 päivää, joten huhtikuulta korkopäiviä kertyy 30-2 = 28 kpl.
            \item Toukokuulta päiviä kertyy 31
            \item Kesäkuulta päiviä kertyy 16
        \end{itemize}
        Yhteensä korkopäiviä kertyy 75, joten koron suuruus on
        \[
            r = kit = 800\cdot 0,063\cdot \frac{75}{365} = 10,36
        \]
        euroa. Jos kyseessä on karkausvuosi, koron suuruus on
        \[
            r = kit = 800\cdot 0,063\cdot \frac{75}{366} = 10,33
        \]
        euroa. Tämä ei ole ainoa tapa laskea korkoajan osuutta vuodesta.
    \end{ratkaisu}
\end{frame}

\begin{frame}
    \frametitle{Yksinkertainen korkolasku}
    \begin{block}{Korkoajan laskeminen (3 tapaa)}
        \begin{itemize}
            \item \emph{Todelliset/365}: todellinen päivien lkm, vuodessa 365 (366) päivää
            \item \emph{Todelliset/360}: todellinen päivien lkm, vuodessa 360 päivää
            \item \emph{30/360}: kuukaudessa 30 päivää, vuodessa 360 päivää
        \end{itemize}
    \end{block}
    \begin{figure}
        \includegraphics[scale=0.7]{months}
    \end{figure}
\end{frame}

\begin{frame}
    \frametitle{Yksinkertainen korkolasku}
    \begin{esim}
        Tilille, jonka korkokanta on 0,75 \% p.a., talletetaan tammikuun 13. päivä 700 euroa. \pause
        Laske kertyneen koron suuruus samana vuonna
        \begin{enumerate}[(a)]
            \item 13.5.
            \item 7.9.
        \end{enumerate}
        Käytä yleistä laskutapaa todelliset/360.
    \end{esim}
\end{frame}

\begin{frame}
    \begin{ratkaisu}
        \begin{enumerate}[(a)]
            \item
                \begin{itemize}
                    \item Tammikuussa päiviä kertyy 31-13 = 18 kpl.
                    \item Helmikuussa päiviä on 28, maaliskuussa 31, huhtikuussa 30.
                    \item Toukokuulta päiviä kertyy 13.
                \end{itemize}
            Yhteensä korkopäiviä kertyy siis 120 kpl. Koron suuruus on lähdevero huomioiden
            \[
                r = kit = 700\cdot 0,70\cdot0,0075\cdot\frac{120}{360} \approx 1,23
            \]
            euroa.
        \end{enumerate}
    \end{ratkaisu}
\end{frame}

\begin{frame}
    \begin{ratkaisu}
        \begin{itemize}
            \item[(b)]
                \begin{itemize}
                    \item Tammikuussa päiviä kertyy 31-13 = 18 kpl.
                    \item Helmikuussa päiviä on 28, maaliskuussa 31, huhtikuussa 30, toukokuussa 31, kesäkuussa 30
                    \item Heinäkuussa ja elokuussa päiviä on 31
                    \item Syyskuulta päiviä kertyy 7
                \end{itemize}
            Yhteensä korkopäiviä kertyy siis 237 kpl. Koron suuruus on lähdevero huomioiden
            \[
                r = kit = 700\cdot 0,70\cdot0,0075\cdot\frac{237}{360} \approx 2,42
            \]
            euroa.
        \end{itemize}
    \end{ratkaisu}
\end{frame}

\begin{frame}
    \frametitle{Aritmeettinen jono ja summa}
    \begin{block}{Aritmeettinen jono}
        Lukujono \pause on \emph{aritmeettinen}, jos sen peräkkäisten jäsenten erotus (etäisyys) on vakio.
    \end{block}
    \pause
    \begin{block}{Aritmeettinen summa}
        \emph{Aritmeettinen summa} \pause on aritmeettisen lukujonon ($n$ ensimmäisen jäsenen) summa. \pause Se saadaan kaavasta
        \[
            S_n = n \cdot \frac{a_1 + a_n}{2}\,,
        \]
        \pause missä $n$ on yhteenlaskettavien lukumäärä, \pause $a_1$ summan ensimmäinen termi ja \pause $a_n$ summan viimeinen termi.
    \end{block}
    \pause
    Aritmeettisen summan kaava on oivallinen työkalu, kun tarkastellaan toistuvia talletuksia yksinkertaisen korkolaskennan puitteissa.
\end{frame}

\begin{frame}
    \frametitle{Yksinkertainen korko - toistuvat talletukset}
    \begin{esim}
        Tilille, jonka korkokanta on 1,5 \% p.a, talletetaan maaliskuusta alkaen 100 euroa kunkin kuukauden lopussa.
        Viimeinen talletus tehdään joulukuun lopussa, jonka jälkeen tilille maksetaan korko. Laske koron ja nettokoron suuruus.
    \end{esim}
\end{frame}

\begin{frame}
    \frametitle{Yksinkertainen korko}
	\begin{ratkaisu}
		\begin{itemize}
			\item Maaliskuun talletus kasvaa korkoa huhti--joulukuun eli 9 kk
			\item Huhtikuun talletus kasvaa korkoa 8 kk
			\item Toukokuun talletus kasvaa korkoa 7 kk jne.
			\item Marraskuun talletus kasvaa korkoa yhden kuukauden
			\item Joulukuun talletus ei kasva korkoa
		\end{itemize}
	\end{ratkaisu}
\end{frame}

\begin{frame}
\frametitle{Yksinkertainen korko}
	\begin{ratkaisu}		
		Edellisen perusteella koron suuruus on 
		\begin{align*}
			&100\cdot0,015\cdot\frac{8}{12}+100\cdot0,015\cdot\frac{7}{12}+\ldots+100\cdot0,015\cdot\frac{1}{12} \\
			&= 100\cdot0,015\cdot\left(\frac{9}{12} + \frac{8}{12} + \cdots +\frac{1}{12}\right)\\
			&= 100\cdot0,015\cdot\frac{1}{12}\cdot(9+8+\cdots+1)\\
			&= \frac{1,5}{12}\cdot(9+8+\cdots+1)\\
			&= \frac{1,5}{12}\cdot\left( 9\cdot\frac{9+1}{2}\right) = 5,625
		\end{align*}
		euroa. Lähdeveron jälkeen nettokoroksi jää \(0,70\cdot5,625=3,94\) euroa.
	\end{ratkaisu}
\end{frame}

\begin{frame}
\frametitle{Koronkorko}
	\begin{esim}
		Säästötilille, jonka korkokanta on 1,5~\% p.a. talletetaan 1500 euroa.
		Kuinka paljon tililtä on nostettavissa rahaa viiden vuoden kuluttua, jos lähdevero
		\begin{enumerate}[(a)]
			\item jätetään huomiotta
			\item huomioidaan?
		\end{enumerate}
	\end{esim}
\end{frame}

\begin{frame}
\frametitle{Koronkorko}
	\begin{block}{Kasvanut pääoma koronkoron tapauksessa}
		\[
			K = kq^n
		\]
		Kasvanut pääoma \(K\) riippuu alkuperäisestä pääomasta \(k\), korkotekijästä \(q\) ja korkokausien lukumäärästä \(n\)
	\end{block}
	Yllä korkotekijä (tai korkokerroin) \(q\) on korkokannasta saatava prosenttikerroin.
\end{frame}

\begin{frame}
 	\begin{ratkaisu}
 		\begin{enumerate}[(a)]
 			\item Jos lähdevero jätetään huomiotta, korkotekijä on \(1+ 0,015 = 1,015\) ja kasvanut pääoma
 			\(K=1500\cdot1,015^5\approx 1615,93\) euroa.
 			\item Jos lähdevero huomioidaan, korkotekijä on \(1+0,7\cdot0,015=	1,0105\). Silloin kasvanut pääoma on
 			\(K=1500\cdot1,0105^5\approx	1580,42\) euroa.
 		\end{enumerate}
 	\end{ratkaisu}
\end{frame}

\begin{frame}
    \frametitle{Geometrinen jono ja summa}
    \pause
    \begin{block}{Geometrinen jono}
        Lukujono on \emph{geometrinen}, jos sen peräkkäisten jäsenten suhde (osamäärä) on vakio.
    \end{block}
    \pause
    \begin{block}{Geometrinen summa}
        \emph{Geometrinen summa} tarkoittaa geometrisen jonon ($n$ ensimmäisen jäsenen) summaa. Se saadaan kaavasta
        \[
            S_n = a_1\cdot \frac{1-q^n}{1-q}, \qquad q\neq 1,
        \] \pause
        missä \(n\) on summattavien lukumäärä,
        \pause \(a_1\) ensimmäinen summattava
        \pause ja \(q\) suhdeluku.
    \end{block}
\end{frame}

\begin{frame}
\frametitle{Koronkorko: toistuvat talletukset}
	\begin{esim}
		Säästötilille, jonka korkokanta on 2,7 \% p.a. talletetaan 900 euroa vuosittain aina vuoden alussa.
		Kuinka paljon tililtä on nostettavissa rahaa viiden vuoden kuluttua ensimmäisestä talletuksesta?
		Kuinka paljon korkoa on kertynyt yhteensä?
	\end{esim}
\end{frame}

\begin{frame}
	\begin{ratkaisu}
		Korkokanta on 0,027, joten nettokorkokanta on \(0,7\cdot0,027 = 0,0189\).
		Siis korkotekijä on \(1+0,7\cdot0,027 =	1,0189\).
		\begin{itemize}
		    \item Ensimmäinen talletus kasvaa korkoa 5 vuotta, eli tulee \(1,0189^5\)-kertaiseksi\pause
		    \item Toinen talletus kasvaa korkoa 4 kertaa, eli tulee \(1,0189^4\)-kertaiseksi\pause
		    \item Kolmas talletus kasvaa korkoa 3 kertaa, eli tulee \(1,0189^3\)-kertaiseksi\pause
		    \item Neljäs talletus kasvaa korkoa 2 kertaa, eli tulee \(1,0189^2\)-kertaiseksi\pause
		    \item Viides talletus kasvaa korkoa yhden kerran, eli tulee \(1,0189\)-kertaiseksi
		\end{itemize}
	\end{ratkaisu}
\end{frame}


\begin{frame}
    \begin{ratkaisu}
        Kaiken kaikkiaan viiden vuoden kuluttua tilillä on rahaa
        \begin{multline*}
            900\cdot1,0189^5 + 900\cdot1,0189^4 + 900\cdot1,0189^3\\
             +900\cdot1,0189^2 + 900\cdot1,0189
        \end{multline*}
        euroa, eli\pause
        \begin{align*}
            &900(1,0189^5 + 1,0189^4 + 1,0189^3 + 1,0189^2 + 1,0189)\\
            =&900(1,0189 + 1,0189^2 + 1,0189^3 + 1,0189^4 + 1,0189^5)\\
            =&900\cdot1,0189\cdot\frac{1-1,0189^5}{1-1,0189}\\
            \approx & 4761,67
        \end{align*}
        euroa. \pause Korkoa on saatu \(4761,68-5\cdot900 = 261,68\) euroa.
    \end{ratkaisu}
\end{frame}

\begin{frame}
    \frametitle{Koronkorkolaskenta: toistuvat talletukset}
    \pause
    \begin{esim}
        Avaat vuoden alussa tilin, jolle talletat joka kuukauden alussa 100 euroa.
        Tilin korkokanta on 2 \%. Kuinka paljon rahaa tilillä on kolmen vuoden kuluttua korkojen lisäämisen jälkeen?
        Kuinka paljon korkoa on kertynyt yhteensä? Huomioi 30 \% lähdevero.
    \end{esim}
\end{frame}

\begin{frame}
    \frametitle{Koronkorkolaskenta: toistuvat talletukset}
    \pause
    \begin{ratkaisu}
        Tarkastellaan ensin yhden vuoden tilannetta. Tammikuussa talletetun summan korkoaika on 12/12 vuotta, helmikuussa
        talletetun summan 11/12 vuotta jne. Nettokorkokanta on \(0,70\cdot 0,02 = 0,014\).
        Vuoden lopussa tilille maksettava korko on siis \pause
        \begin{align*}
             &100\cdot0,014\cdot\frac{12}{12} + 100\cdot0,014\cdot\frac{11}{12} + \ldots + 100\cdot0,014\cdot\frac{1}{12}\\
            =&100\cdot0,014\cdot\frac{1}{12}\cdot(12+11+\ldots+1) = 100\cdot0,014\cdot\frac{1}{12}\cdot78\\
            =&9,10
        \end{align*}
        euroa. Tilille vuosittain talletettava summa 1200 euroa kasvaa siis korkoa 9,10 euroa samana vuonna.
    \end{ratkaisu}
\end{frame}

\begin{frame}
    \begin{ratkaisu}
        Kukin näin laskettu vuosittainen summa kasvaa seuraavan kerran korkoa vasta seuraavana vuonna.
        \pause
        Näin ollen tilillä on loppujen lopuksi, eli kolmen vuoden kuluttua ensimmäisestä talletuksesta, rahaa \pause
        \[
            1209,10\cdot 1,014^2+1209,10\cdot1,014+1209,10\approx 3678,32
        \]
        euroa. Kaiken kaikkiaan tilille on talletettu \(3\cdot12\cdot100 = 3600\) euroa. Korkoa on siis kertynyt yhteensä 78,32 euroa.
    \end{ratkaisu}
\end{frame}

\begin{frame}
    \frametitle{Diskonttaus}
    \pause
    \begin{esim}
        Kuinka paljon pitäisi tallettaa pankkitilille, jotta neljän vuoden kuluttua siltä olisi nostettavissa 1 000 euroa? 
        Tilin vuotuinen korko on 4 \%. Ratkaise tehtävä
        \begin{enumerate}
            \item[(a)] olettaen, että lähdeveroa ei peritä.
            \item[(b)] lähdevero huomioiden.
        \end{enumerate}
    \end{esim}
    \pause
    Korkokausien aikana kertyneen koron poistamista pääoman arvosta eli alkuperäisen pääoman selvittämistä kutsutaan 
    \emph{diskonttaukseksi}.
    %Tähän liittyi myös yksinkertaisen korkolaskennan esimerkki aiemmin.
    \pause
    Diskonttauksella saatua alkuperäistä pääomaa nimitetään \emph{nykyarvoksi}.
\end{frame}

\begin{frame}
    \begin{ratkaisu}
        Olkoon alkuperäinen pääoma \(k\).
        \begin{enumerate}[(a)]
            \item Korkokerroin on \(1,04\), joten neljän vuoden kuluttua tililtä on nostettavissa \(1,04^4k\) euroa.
                Nyt
                \[
                    1,04^4k = 1000\quad\Leftrightarrow\quad k = \frac{1000}{1,04^4}\approx 854,80
                \]
                euroa.
            \item Nettokorkokanta on \(0,70\cdot0,04=0,028\), joten korkokerroin on \(1,028\) kun lähdevero otetaan huomioon.
                Jotta tilillä olisi neljän vuoden kuluttua 1000 euroa, täytyy olla
                \[
                    1,028^4\cdot k = 1000\quad\Leftrightarrow\quad k = \frac{1000}{1,028^k}\approx 895,42
                \]
                euroa.
        \end{enumerate}
    \end{ratkaisu}
\end{frame}

\begin{frame}
    \frametitle{Diskonttaus: nykyarvo koronkoron tapauksessa}
    \pause
    \begin{block}{}
        \emph{Nykyarvo} eli alkuperäinen pääoma $k$ koronkoron tapauksessa on
        \[
            k = Kq^{-n} = \frac{K}{q^n},
        \]
        missä $K$ on kasvanut pääoma, $q$ on korkotekijä ja $n \geq 1$ on korkokausien lukumäärä.
        \pblock
        Kerroin $q^{-n} = \dfrac{1}{q^n}$ on nimeltään \emph{diskonttaustekijä}.
    \end{block}
\end{frame}

\begin{frame}
    \frametitle{Investointilaskelmia nykyarvomenetelmällä}
    \begin{itemize}
        \item {Investointi} tarkoittaa välineiden tai maan hankkimista tuotantoa tai toimintaa varten. \pause
        \item Investoinnin kannattavuuden arvioimiseen voidaan käyttää \emph{nykyarvomenetelmää}, jossa kaikki menot ja tulot diskontataan investoinnin alkuhetkeen.  \pause
        \item Investointi on kannattava, jos tulot ovat suuremmat kuin menot. \pause
        \item Diskonttauksessa käytetty korkokanta voi määräytyä esimerkiksi yrityksen omista tuottovaatimuksista tai pankin korkokannasta.
    \end{itemize}
\end{frame}

\begin{frame}
    \frametitle{Investointilaskelmia nykyarvomenetelmällä}
    Investointilaskelmiin liittyviä peruskäsitteitä ovat \pause
    \begin{itemize}
        \item \emph{Perushankintakustannus}: investoinnin alkuun liittyvä kertakustannus. \pause
        \item \emph{Investointiaika}: aika, jolloin investoinnista oletetaan saatavan hyötyä. \pause
        \item \emph{Jäännösarvo}: investoinnin arvo investointiajan lopussa.
    \end{itemize}
\end{frame}

\begin{frame}
\frametitle{Investointilaskelmia nykyarvomenetelmällä}
%KEVENNÄ TARVITTAESSA VÄHENTÄMÄLLÄ VUOSIEN MÄÄRÄÄ
    \begin{esim}
    Yritys harkitsee uusien monitoimikopiokoneiden hankkimista.
    Kopiokoneiden yhteishinta on 6 000 euroa. Niiden käyttöiäksi on arvioitu 5 vuotta ja jälleenmyyntiarvoksi 5 \% hankintahinnasta.
    Kopiokoneiden arvellaan vähentävän kustannuksia kolmena ensimmäisenä vuonna 1 500 euroa vuodessa ja kahtena viimeisenä vuonna 1 000 euroa vuodessa.
    \pblock
    Onko kopiokoneiden hankkiminen yritykselle kannattavaa, jos hankinta rahoitetaan lainalla, jonka vuosikorko on 3 \%?
    \end{esim}
\end{frame}

\begin{frame}
    \begin{ratkaisu}
        Säästöt olisivat nykyhetkessä
        \[
            \frac{1500}{1,03} + \frac{1500}{1,03^2} + \frac{1500}{1,03^3} + \frac{1000}{1,03^4} + \frac{1000}{1,03^5}\approx 5994,01
       \]
       euroa ja jälleenmyyntiarvo
       \[
            0,05\cdot\frac{6000}{1,03^5}\approx 258,78
       \]
       euroa. Tuotot olisivat nykyhetkessä yhteensä 6252,80 euroa. Koska lainaa tarvitsee ottaa vain 6000 euroa,
       investointi on tämän menetelmän valossa kannattava.
    \end{ratkaisu}
\end{frame}

\begin{frame}
    \frametitle{Nelilaskintekniikkaa: eksponentin ratkaiseminen}
    \pause
    \begin{esim}
        Missä ajassa tilille talletettu 500 euroa on kasvanut 600 euroksi, jos tilin korkokanta on 4 \% p.a.?
        Ratkaise tehtävä
        \begin{enumerate}[(a)]
            \item olettaen, että lähdeveroa ei peritä.
            \item 30 \% lähdevero huomioiden.
        \end{enumerate}
    \end{esim}
    \pause
    Kysytyn eksponentin  eli korkokausien määrän $n$ voi selvittää kokeilemalla!
\end{frame}

\begin{frame}
    \frametitle{Nelilaskintekniikkaa: kantaluvun selvittäminen}
    \begin{esim}
        Talletit säästötilille 5 000 euroa. Kolmen vuoden kuluttua nostit tilisi tyhjäksi ja sait 5 950 euroa.
        Oletetaan, että tilillä ei ollut koronmaksun lisäksi muita tilitapahtumia. Mikä oli tilin
        \begin{enumerate}[(a)]
            \item nettokorkokanta
            \item bruttokorkokanta?
        \end{enumerate}
    \end{esim}
    \pause
    Myös kantaluvun eli korkotekijän $q$ voi selvittää kokeilemalla!
\end{frame}

\end{document}

TODO: Diskonttaus kumminkin päin

\begin{frame}
    \frametitle{Yksinkertainen korkolaskenta: lähdeveron pyöristyssääntö}
    \begin{ratkaisu}
        Talletat huhtikuun alussa 100 000 euroa tilille, jonka korkokanta on 2{,}6~\%~p.a.
        Kuinka paljon voit nostaa tililtäsi 100 päivän kuluttua? Huomioi 30 \% lähdevero ja sen pyöristyssääntö.
    \end{ratkaisu}
\end{frame}

\begin{frame}
    \frametitle{Koronkorkolaskenta: lähdeveron huomioiminen}
    \pause
    \bigskip
    Lähdevero 30 \% peritään korosta vuosittain, joten kertyneestä korosta vain 70 \% liitetään pääomaan vuosittain.
    Kasvanut pääoma saadaan laskettua käyttämällä \emph{nettokorkokantaa}, jossa lähdeveron vaikutus on huomioitu.
    \pause
    Edellisessä esimerkissä nettokorkokanta on $0{,}7i = 0{,}7 \cdot 0{,}035 = 0{,}0245$ ja korkotekijä $q = 1 + 0{,}7i = 1{,}0245$.
\end{frame}